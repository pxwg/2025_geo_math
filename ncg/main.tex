% !TeX program = xelatex
\documentclass[10pt]{article}
\input{../preamble.tex}
\fancyhf{}
% ltex: enabled=false
\fancypagestyle{plain}{
  \lhead{2025}
  \chead{\centering{Introduction to Noncommutative Geometry}}
  \rhead{\thepage\ of \pageref{LastPage}}
  \lfoot{}
  \cfoot{}
\rfoot{}}
\pagestyle{plain}
%-------------------fonts-------------------
\usepackage[sc]{mathpazo}
\usepackage{courier}
% \usepackage[utf8]{inputenc}
\usepackage[T1]{fontenc}
\usepackage{microtype}
\RequirePackage[font=small,format=plain,labelfont=bf,textfont=it]{caption}
% ltex: enabled=true

%-------------------basic info-------------------

\title{\textbf{Introduction to Noncommutative Geometry}}
\author{Xinyu Xiang}
\date{Jul. 2025}

%-------------------document---------------------

\begin{document}
\maketitle

\section{Day I: Banach Space, Hilbert Space and C-star Algebra}

\subsection{Banach Space and Hilbert Space}
\begin{equation*}
  \begin{aligned}
    \text{Hilbert Space } \mathcal{H} & \longleftrightarrow \text{Banach Space $\mathcal{B}$ + Norm from Inner Product} \\
    & \longleftrightarrow \text{Vector Space + Inner Product + Complete Norm},
  \end{aligned}
\end{equation*}
where one have unique inner product $\left< ~,~ \right>$ from the norm $\| ~ \|$
\begin{equation*}
  \left< \phi, \psi \right> =
\end{equation*}

\subsection{C-star Algebra}

\begin{definition}[Algebra]
  An \textbf{algebra} $\mathcal{A}$ over a field $\mathbb{F}$ is a vector space over $\mathbb{F}$ equipped with a bilinear product $\mathcal{A} \times \mathcal{A} \to \mathcal{A}$, denoted by $a \cdot b$ or simply $ab$.
\end{definition}

\begin{definition}[Banach Algebra]
  A Banach algebra is an algebra $ \mathcal{A}$ endowed with a norm $\| \cdot \|$ such that
  \begin{equation*}
    \| ab \| \leq \| a \| \| b \|, \quad \forall a, b \in \mathcal{A}.
  \end{equation*}
\end{definition}

\begin{definition}[$ \mathbb{C}^{*}$ Algebra]
  $ \mathbb{C}^{*}$ algebra $\longleftrightarrow$ Banach algebra over $ \mathbb{C}$ + $\forall x \in \mathbb{C}^{*}$, $ \phi: x \mapsto x^{*}$.
\end{definition}

\begin{example}[Matrix]

\end{example}
\begin{example}[Continues Linear Operator over Hilbert Space]

\end{example}

\label{LastPage}
\end{document}
