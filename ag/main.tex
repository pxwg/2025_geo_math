% !TeX program = pdflatex
\documentclass[10pt]{article}
\input{../preamble.tex}
\fancyhf{}
\fancypagestyle{plain}{
  \lhead{2025}
  \chead{\centering{Algebraic Curve}}
  \rhead{\thepage\ of \pageref{LastPage}}
  \lfoot{}
  \cfoot{}
\rfoot{}}
\pagestyle{plain}

%-------------------basic info-------------------

\title{\textbf{Algebraic Curve}}
\author{Lectured by Wenchuan Hu and Noted by Xinyu Xiang}
\date{Jun. 2025}

%-------------------document---------------------

\begin{document}
\maketitle

% Fields and 代数闭域:Z, Q, F_p
% AIM: Consider Bezout Theorem and RR Theorem
\section{Day 0: Preliminary}

\section{Day I}

% \subsection{Very Very Briefly Introduction to Bezout Theorem}
%
% Consider the following polynomial equation:
% \begin{equation*}
%   \begin{cases}
%     y^{2} = x^{3} - x, \, \deg = 3\\
%     x = 1, \, \deg = 1.
%   \end{cases}
% \end{equation*}
% Solve the equation above, one could find that the solution is $ (1, 0)$, which could be interpreted as the intersection of the curve $ y = x^{3} - x$ and $ x = 1$.

\begin{definition}[Polynomial]
  The collection of polynomials would denoted by $ \mathbb{K}[x_1,\cdots ,x_{n}]$, whose elements are of the form
  \begin{equation*}
    f = \sum_{i_1,\cdots ,i_n} a_{i_1,\cdots ,i_n} x_1^{i_1}\cdots x_n^{i_n},
  \end{equation*}
  where $ a_{i_1,\cdots ,i_n} \in \mathbb{K}$, and $ i_1,\cdots ,i_n$ are non-negative integers.
\end{definition}

\begin{definition}[Algebraic Closed Field]
  If
\end{definition}

\begin{remark}
  Finite field is not algebraic closed: Consider $ f = (x - a_{1}) \cdots (x - a_{n}) + 1$ which has no zero point.
\end{remark}

\begin{definition}[Unique Factorization Domain (UFD)]

\end{definition}

\begin{proposition}
  \begin{enumerate}[(1)]
    \item   $ \mathbb{K}[x_1,\cdots ,x_{n}]$ is a commutative ring with unity called the polynomial ring in $ n$ variables over $ \mathbb{K}$.
    \item If $ R$ is UFD, then $ R[X]$ is a UFD, which means that every non-zero polynomial can be factored uniquely into irreducible polynomials, up to order and units.
  \end{enumerate}
\end{proposition}

From here on, we assume that $ \mathbb{K}$ is an algebraic closed field.
\begin{definition}[Affine Variety]
  An affine variety is a subset of $ \mathbb{K}^{n}$ defined by the vanishing of a set of polynomials, i.e., it is the solution set of a system of polynomial equations.

  Formally, given a set of polynomials $ f_1, \ldots, f_m \in \mathbb{K}[x_1,\ldots,x_n]$, the affine variety $ V(f_1, \ldots, f_m)$ is defined as:
  \begin{equation*}
    V(f_1, \ldots, f_m) = \{ (a_1, \ldots, a_n) \in \mathbb{K}^{n} ; f_i(a_1, \ldots, a_n) = 0 \text{ for all } i = 1, \ldots, m \}.
  \end{equation*}
\end{definition}

\begin{proposition}[Zariski Topology]
  Consider $ f, g \in \mathbb{K}[x,y]$
  \begin{enumerate}[(1)]
    \item $V(fg) = V(f) \cup V(g)$ ,
    \item $ V(f, g) = V(f) \cap V(g)$, $ V(f_{\lambda})_{\lambda \in \Lambda} = \bigcap_{\lambda \in \Lambda} V(f_{\lambda})$,
    \item $ V(0) = \mathbb{A}_{\mathbb{K}}^{2}$.
  \end{enumerate}
\end{proposition}

% 平面仿射曲线
\begin{definition}[Affine Curve]
  Consider $ f \in \mathbb{K}[x,y]$, $ V(f)$ denotes affine curve.
  \begin{enumerate}[(1)]
    \item $ \deg V(f) = \deg f$,
      \begin{enumerate}[(a)]
        \item $\deg = 1$: Line,
        \item $\deg = 2$: conic curve (non-degenerate),
      \end{enumerate}
    \item $ F = F_{1}^{n_1} F_2^{n_2} \cdots F_{m}^{n_{m}}$, where $ F_{i}$ irreducible.
  \end{enumerate}
\end{definition}

\begin{example}
  $( x + y )^{2}$ is irreducible, $ xy$ is reducible.
\end{example}
\begin{example}
  $ y^{2} - x^{3} + x$ is irreducible (left as exercise).
\end{example}

% 分式域
\begin{definition}[Field of Fractions]
  The field of fractions of a UFD $ R$ is the smallest field in which $ R$ can be embedded, denoted by $ K(R)$.
  It consists of elements of the form $ \frac{a}{b}$ where $ a, b \in R$ and $ b \neq 0 \in R$.

  Formally, if $ R$ is a UFD, then the field of fractions $ K(R)$ is defined as:
  \begin{equation*}
    Q_{\mathrm{uot}}(R) = \left\{ \frac{a}{b} \mid a, b \in R, b \neq 0 \right\},
  \end{equation*}
  which is indeed a field.
\end{definition}

\begin{lemma}
  Consider $ f \in \mathbb{K}[x,y]$ and $ \deg f > 0$, then
  \begin{enumerate}[(1)]
    \item $ V(f)$ has infinitely many points,
    \item $ \mathbb{A}^{2}_{\mathbb{K}} - V(f)$ has infinitely many points.
  \end{enumerate}
\end{lemma}

% 朱利安整环??
\begin{theorem}[Simple Bezout Theorem]
  If $ F, G \in \mathbb{K}[x,y] \subset \mathbb{K}(x)[y]$ has no common component, then $ V(F,G)$ is a finite set $\Leftrightarrow$ $ F=0$, $ G = 0$ have finite solutions in $ \mathbb{K}^{2}$.
\end{theorem}
\begin{proof}
  \begin{enumerate}[(1)]
    \item Assume there is an element $ \alpha$ such that $ F = \alpha F'$ and $ G = \alpha G'$, where we consider the ring $ \mathbb{K}(x)[y]$, then
      \begin{equation*}
        \begin{cases}
          a F = H F'\\
          b G = H G',
        \end{cases}
      \end{equation*}
      where $ a \in \mathbb{K}[x]$ and $ H \in \mathbb{K}[x,y]$.
    \item TBD
  \end{enumerate}
\end{proof}

\begin{theorem}
  Consider irreducible $ F, G \in \mathbb{K}[x,y]$, $ F | G \Leftrightarrow V(F) \subset V(G)$.
\end{theorem}
\begin{proof}
  \begin{enumerate}[(1)]
    \item If $ F | G$, then $ G = F H$ for some $ H \in \mathbb{K}[x,y]$, thus $ V(F) \subset V(G)$.
    \item If $ V(F) \subset V(G)$, by definition $ F | G$.
  \end{enumerate}
\end{proof}

\section{Day II: Intersection Number (1)}

\begin{definition}[Localized Ring]
  Consider $ \mathbb{K}[x,y]$ and a prime ideal $ P \subset R$, the localized ring $ \mathcal{O}_P$ is defined as:
  \begin{equation*}
    \mathcal{O}_P = \left\{ \frac{f}{g} ; f, g \in \mathbb{K}[x,y] , g(p) \neq 0 \right\},
  \end{equation*}
  the maximal ideal $ \mathfrak{m}_P$ is defined as:
  \begin{equation*}
    \mathfrak{m}_P = \left\{ \frac{f}{g} ; f,g \in \mathbb{K}[x,y] , g(p) \neq 0 , f(p) = 0\right\}.
  \end{equation*}
  which satisfies
  \begin{equation*}
    0 \rightarrow \mathfrak{m}_P \rightarrow \mathcal{O}_P \rightarrow \mathbb{K}.
  \end{equation*}
\end{definition}

\subsection{Definition}

\begin{definition}[Intersection Number]
  Consider $ F, G \in \mathbb{K}[x,y]$ irreducible, $ P = V(F) \cap V(G)$, then the intersection number $ I_P(F,G)$ is defined as:
  \begin{equation*}
    \mu_{P}(F,G) = \dim_{\mathbb{K}} \mathcal{O}_P / \left< F,G \right>,
  \end{equation*}
  where $\left< F,G \right>$ is the ideal generated by $ F$ and $ G$ in the localized ring $ \mathcal{O}_P$.
\end{definition}

\begin{proposition}
  \begin{enumerate}[(1)]
    \item $ \mu_{P}(F,G) \in \mathbb{N} \cup \left\{ \infty  \right\}$;
    \item $ P \in F \cap G \Leftrightarrow \mu_{P}(F,G) \ge 1$, $ \mu_{P}(F,G) = 1 \Leftrightarrow \left< F,G \right> = \mu_{P}$
    \item $ \mu_{P}(F,G) = \mu_{P}(G,F)$;
    \item $ \mu_{P}(F, G+FH) = \mu_{P}(F,G)$;
    \item $ \mu_{P}(FG, H) = \mu_{P}(F,H) + \mu_{P}(G,H);$
    \item $ I_{(0,0)}(x,y) = 1$,
  \end{enumerate}
\end{proposition}

\begin{example}
  Consider $ F = y - x^{2}$ and $ G = y$.
\end{example}
\begin{solution}
  \textbf{Use properties to compute}:
  \begin{equation*}
    \begin{aligned}
      \mu_{0}(y, y - x^{3}) = & \mu_{0}(y , - x^3) \\
      = & 2 \mu_{0} (y,x) \\
      = & 2
    \end{aligned}
  \end{equation*}
  where we used the property (4) to reduce the degree of the polynomial for the given variable, and use the fact that $ \mu_{0}(x,y) = 1$ .
\end{solution}
The most important part is to use the property (4) to reduce the degree of the polynomial for the given variable.
\begin{example}
  Consider $ F = y^{2} - x^{3}$ and $ G = x^{2} - y^{3}$.
\end{example}
\begin{solution}
  \begin{equation*}
    \begin{aligned}
      \mu_{0}(y^{2} - x^{3}, x^{2} - y^{3}) = & \mu_{0}(y^{2} - x^{3} + x (x^{2} - y^{3}), y^{3} - x^{2}) \\
      = & \mu_{0}(y^{2} - x y^{3}, y^{3} - x^{2}) \\
      = & \mu_{0}(y^{2}, y^{3} - x^{2}) + \mu_{0}(1 - x y, y^{3} - x^{2}) \\
      = & 2 \mu_{0}(y, y^{3} - x^{2}) + 0\\
      = & 2 \mu_{0}(y, x^{2})\\
      = & 4 \mu_{0}(y,x) \\
      = & 4\\
    \end{aligned}
  \end{equation*}
  $ \mu_{0}(1-xy, y^{3} - x^{2})$ vanished since at $ (0,0)$, $ 1 - xy \neq 0$ and $ y^{3} - x^{2} = 0$.
\end{solution}

\begin{example}
  Consider $ F = y - x - x^{2}$ and $ G = y^{2} - x^{2} - 3 x^{2}y$.
\end{example}
\begin{solution}
  \begin{equation*}
    \begin{aligned}
      \mu_{0}(y-x- x^{2} , y^{2} - x^{2} - 3 x^{2}y) = & \mu_{0}(y - x - x^{2} , y^{2} - x^{2} - 3 x^{2}y - (x+y)(y - x - x^{2})) \\
      = & \mu_{0}(y - x - x^{2} , - 2 x^{2}y + x^{3}) \\
      = & \mu_{0}(y - x - x^{2}, x^{2} (x - 2y)) \\
      = & 2 \mu_{0}(y - x - x^{2}, x) + \mu_{0}(y - x - x^{2}, x - 2y) \\
      = & 3.
    \end{aligned}
  \end{equation*}
  Another way to compute is to use definition of intersection number, where we plug the equation $ y = x+x^{2}$ into the second equation, we have
  \begin{equation*}
    \mu_{0}(y - x - x^{2} , y^{2} - x^{2} - 3 x^{2}y) = \mathfrak{m}_0\left( (x+x^{2})^{2} - x^{2} - 3 x^{2}(x+x^{2}) \right) = \mathfrak{m}_0 \left( x^{3}(-1-2x) \right) = 0.
  \end{equation*}
\end{solution}

\begin{proposition}
  If the lowest degree of $ F$ is $ x^{n}$ and the lowest degree of $ G$ is $ y^{m}$, then the intersection number $ I_{(0,0)}(F,G)$ is $ nm$.
\end{proposition}

\begin{definition}[Short Exact Sequence]
  A short exact sequence of modules is a sequence of modules and homomorphisms
  \begin{equation*}
    0 \rightarrow A \xrightarrow{f} B \xrightarrow{g} C \rightarrow 0,
  \end{equation*}
  such that the image of $ f$ is equal to the kernel of $ g$, i.e., $ \mathrm{Im}(f) = \ker(g)$.
\end{definition}
We would use the short exact sequence for linear space.
\begin{definition}
  Consider $ P \in \mathbb{A}^{2}$ and $ F, G, H \in \mathbb{K}[x,y]$, then
  \begin{enumerate}[(1)]
    \item If $ F, G$ has no common component cross $ P$, then
      \begin{equation*}
        0 \rightarrow \mathcal{O}_{P}/\left< F,H \right> \xrightarrow{\bullet G} \mathcal{O}_{p}/\left< F, GH \right> \xrightarrow{\pi} \mathcal{O}_{P}/\left< F,G \right> \rightarrow 0,
      \end{equation*}
      where $ \pi$ is the natural projection map.
    \item $ \mu_{P}(F, GH) = \mu_{P}(F,G) + \mu_{P}(F,H)$.
  \end{enumerate}
\end{definition}
% TODO: Prove the above theorem.
\begin{proof}
  \begin{enumerate}[(1)]
    \item $ \pi$ is surjection;
    \item Consider an element acted by multiplication by $ G$:
      \begin{equation*}
        \bullet G: \frac{f}{g} + a F + b H \mapsto F(a G) + G\left( \frac{f}{g} + b H \right) \in \ker{\pi},
      \end{equation*}
      where $ a, b \in \mathbb{K}[x,y]$ and $ g \in \mathcal{O}_{P}$. On the other side, consider $ f/g \in \ker{\pi}$, thus $ f / g = a F + b G \rightarrow b \in \mathcal{O}_{p} / \left< F,H \right>$.
    \item $\bullet G$ is injection.
  \end{enumerate}
  Note that all the vector spaces are finite dimensional, thus the dimension of the kernel is equal to the dimension of the image, and we can conclude that
  \begin{equation*}
    \mu_{P}(F, G H) = \mu_{P}(F,G) + \mu_{P}(F,H),
  \end{equation*}
  which proofs the proposition (5).
\end{proof}

\subsection{The Algorithm to Compute Intersection Number}

Consider $ F(x,y) \in \mathbb{K}[x,y]$, in order to compute the insertion number $ \mu_{0}(y, F(x,y))$, we can expand $ F$ as $ F(x,y) = F(x,0) + y H(x,y)$, thus
\begin{equation*}
  \mu_{0}(y, F(x,y)) = \mu_{0}(y, F(x,0) + y H(x,y)) = \mu_{0}(y, F(x,0)).
\end{equation*}
Assume $ F(x,0) = x^{m} f(x)$ where $ f(x)$ is no vanishing at $ x=0$, thus
\begin{equation*}
  \mu_{0}(y, F(x,y)) = m.
\end{equation*}

Now we shell consider the linear (homogeneous 1 degree part). We denote $ F \in \mathbb{K}[x,y]$ as
\begin{equation*}
  F = F_0 + F_1 +\cdots
\end{equation*}
where $ F_{i}$ is homogeneous degree $ i$ part. The $ F_{1}$ part is important, because of the theorem below:
\begin{theorem}[2.17 Intersection multiplicity 1]
  If $ F, G \in \mathbb{K}[x,y]$ pass through the origin, then
  \begin{equation*}
    \mu_{0}(F,G) = 1 \Leftrightarrow F, G \text{ Linear Independent}
  \end{equation*}
\end{theorem}

% scanned and generated by gpt-4o
\begin{definition}[Tangents and multiplicities of points]
  Let $ F \in \mathbb{K}[x,y]$ be a curve, then
  \begin{enumerate}[(1)]
    \item The smallest \( m \in \mathbb{N} \) for which the homogeneous part \( F_m \) is non-zero is called the multiplicity \( m_0(F) \) of \( F \) at the origin. Any linear factor of \( F_m \) (considered as a curve) is called a tangent to \( F \) at the origin.
    \item For a general point \( P = (x_0, y_0) \in \mathbb{A}^2 \), tangents at \( P \) and the multiplicity \( m_P(F) \) are defined by first shifting coordinates to \( x' = x - x_0 \) and \( y' = y - y_0 \), and then applying (a) to the origin \( (x', y') = (0, 0) \).
  \end{enumerate}
\end{definition}

\section{Day III: Intersection Number (2)}

\begin{definition}[Cusps]
  Let \( P \) be a point on an affine curve \( F \). We say that \( P \) is a cusp if \( m_P(F) = 2 \), there is exactly one tangent \( L \) to \( F \) at \( P \), and \( \mu_P(F,L) = 3 \).
\end{definition}

\begin{definition}[Singular Curve and Non-singular Curve]
  An affine curve $ F \in \mathbb{K}[x,y]$ is called singular if it has a point $ P$ such that $ \mu_{P}(F) > 1$ . If $ F$ has no point $ P$ such that $ \mu_{P}(F) > 1$, then $ F$ is called non-singular.
\end{definition}
where the multiplicity $ \mu_{P}(F)$ is defined as the number of tangents at $ P$.
\begin{proposition}[Affine Jacobi Criterion]
  Let \( P = (x_0, y_0) \) be a point on an affine curve \( F \).
  \begin{enumerate}
    \item[(a)] \( P \) is a singular point of \( F \) if and only if
      \begin{equation*}
        \frac{\partial F}{\partial x}(P) = \frac{\partial F}{\partial y}(P) = 0,
      \end{equation*}
    \item[(b)] If \( P \) is a smooth point of \( F \), the tangent to \( F \) at \( P \) is given by
      \begin{equation*}
        T_P F = \frac{\partial F}{\partial x}(P) \cdot (x - x_0) + \frac{\partial F}{\partial y}(P) \cdot (y - y_0).
      \end{equation*}
  \end{enumerate}
\end{proposition}

\begin{example}
  Consider the tangent $ T_{P}F$ of the curve $ F \in \mathbb{K}[x,y]$, compute the intersection number $ \mu_{P}(F,T_{P}F)$.
\end{example}
\begin{solution}
  First, one can consider some basic examples. For example, consider $ F = y - x^{2}$, thus the tangent at $ P = (0,0)$ is $ T_{P}F = y$, so that the intersection number is
  \begin{equation*}
    \mu_{0}(y, y - x^{2}) = 2.
  \end{equation*}
  Moreover, one can prove that $ \mu_{P}(T_{P}F, F) = 2$ for $ F = y - x^{2}$.
\end{solution}

\begin{theorem}
  Let $ P$  be a smooth point on a curve $ F$ . Then for any two curves $G$  and $H$  that do not have a common component with $F$ through $P$  we have
  \begin{equation*}
    \langle F,G\rangle\subset\langle F,H\rangle\text{ in }\mathscr{O}_P\quad\Leftrightarrow\quad\mu_P(F,G)\geq\mu_P(F,H).
  \end{equation*}
\end{theorem}

\section{Day IV: Projective Curve}

\begin{definition}[Projective Space]
  For $ n \in \mathbb{N}$, we define the projective space $ \mathbb{P}^{n}(\mathbb{K})$ as the set of equivalence classes of non-zero vectors in $ \mathbb{K}^{n+1}$, where two vectors $ (x_0, x_1, \ldots, x_n)$ and $ (y_0, y_1, \ldots, y_n)$ are equivalent if there exists a non-zero scalar $ \lambda \in \mathbb{K}$ such that
  \begin{equation*}
    \sim : (x_0, x_1, \ldots, x_n) = \lambda (y_0, y_1, \ldots, y_n).
  \end{equation*}
  The projective space could thus be defined as:
  \begin{equation*}
    \mathbb{P}^{n} = \left\{ \mathbb{A}_{\mathbb{K}}^{n+1} - \left\{ 0 \right\} \right\}/\sim .
  \end{equation*}
\end{definition}

\begin{example}
  Consider the projective space $ \mathbb{C}P^{2} = \mathbb{C}^{3} - \left\{ 0 \right\} / \sim $, one would induce the fiberation:
  \begin{equation*}
    \mathbb{S}^{1} \rightarrow \mathbb{S}^{5} \xrightarrow{\pi} \mathbb{C}P^{2}.
  \end{equation*}
\end{example}

\begin{example}
  Consider the curve $ F = y - x^{2}$, in $ \mathbb{P}^{2}$ we can introduce the homogeneous coordinate $[x:y:z]$, thus the curve can be written as:
  \begin{equation*}
    F = yz - x^{2},
  \end{equation*}
  while $ z = 0$ (the point at infinity), we have $[0:1:0]$, which is the point at infinity of the curve $ F$.
\end{example}

\label{LastPage}
\end{document}
