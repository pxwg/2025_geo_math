% !TeX program = pdflatex
\documentclass[10pt]{article}
%% ======================================================================
%% LaTeX Preamble for Physics Note, Experiment Report, and Research in
%% CJK and English
%% ======================================================================

%% ----------------------------------------------------------------------
%% 1. 文档基础设置与语言支持
%% ----------------------------------------------------------------------
\usepackage{geometry}
\geometry{a4paper,centering,scale=0.75} % 页面设置

% 中文支持 - 适配 macOS
% \usepackage[UTF8]{ctex}
% \AtBeginDocument{\small} % 全局缩小字体
% \setCJKmainfont[
%   Script=CJK,
%   BoldFont={Heiti SC},
%   ItalicFont={Kaiti SC}
% ]{Songti SC}

%% ----------------------------------------------------------------------
%% 2. 数学支持包
%% ----------------------------------------------------------------------
\usepackage{amsmath,amssymb,amsthm} % 数学基础支持
\usepackage{mathtools} % amsmath 扩展
\usepackage{physics} % 物理公式简化
\usepackage{mathrsfs} % 数学花体字
\usepackage{bbm} % 黑板粗体
\usepackage{cancel} % 公式划除线
\numberwithin{equation}{section} % 按节编号公式
\usepackage[sc]{mathpazo}
\usepackage{courier}
\usepackage{inputenc}
\usepackage[T1]{fontenc}
\usepackage{microtype}
\RequirePackage[font=small,format=plain,labelfont=bf,textfont=it]{caption}

%% ----------------------------------------------------------------------
%% 3. 图形与绘图支持
%% ----------------------------------------------------------------------
\usepackage{graphicx} % 图片支持
\usepackage[export]{adjustbox} % 图片调整
\usepackage{float} % 控制浮动体
\usepackage{pgfplots} % 绘图支持
\pgfplotsset{compat=newest} % 使用最新版本特性
% \usepackage[compat=1.1.0,warn luatex=false]{tikz-feynman} % Feynman 图
\usepackage{quiver} % 交换图

%% ----------------------------------------------------------------------
%% 4. 表格支持
%% ----------------------------------------------------------------------
\usepackage{tabularx} % 增强表格
\usepackage{xltabular} % 长表格支持

%% ----------------------------------------------------------------------
%% 5. 交叉引用与超链接
%% ----------------------------------------------------------------------
% hyperref 通常放在最后加载以避免冲突
\usepackage[bookmarks=true, colorlinks=true, linkcolor=teal,
citecolor=blue, urlcolor=magenta, hidelinks]{hyperref}

%% ----------------------------------------------------------------------
%% 6. 外观与装饰
%% ----------------------------------------------------------------------
\usepackage{fancyhdr} % 页眉页脚
\usepackage[dvipsnames,svgnames]{xcolor} % 扩展颜色支持
\usepackage{framed} % 框架效果
\usepackage{tcolorbox} % 彩色文本框
\tcbuselibrary{most} % tcolorbox 扩展库
\usepackage[strict]{changepage} % 提供 adjustwidth 环境
\usepackage{scalerel} % 缩放支持

%% ----------------------------------------------------------------------
%% 7. 引用与杂项支持
%% ----------------------------------------------------------------------
\usepackage{enumerate} % 列表环境
\usepackage{stackrel} % 符号堆叠
\usepackage{import,xifthen,pdfpages} % 文档导入与条件判断

% 条件加载透明效果包 (仅在 PDF 模式下)
\usepackage{ifpdf}
\ifpdf
\usepackage{transparent}
\else
% 非 PDF 模式下使用替代方案或提供警告
\newcommand{\transparent}[1]{}
\typeout{警告:transparent 包功能在非 PDF 模式下不可用}
\fi

%% ----------------------------------------------------------------------
%% 8. 定理环境设置
%% ----------------------------------------------------------------------
% English theorem environments
\newtheorem{theorem}{Theorem}[section]
\newtheorem{lemma}[theorem]{Lemma}
\newtheorem{proposition}[theorem]{Proposition}
\newtheorem{corollary}[theorem]{Corollary}
\newtheorem{definition}{Definition}[section]
\newtheorem{remark}{Remark}[section]
\newtheorem{example}{Example}[section]
\newtheorem{construction}{Construction}[section]
\newenvironment{observation}{
\begin{proof}[Observation]}{
\end{proof}}
\newenvironment{solution}{
\begin{proof}[Solution]}{
\end{proof}}

%% ----------------------------------------------------------------------
%% 9. 自定义框架和环境
%% ----------------------------------------------------------------------
% 引用块样式
\definecolor{formalshade}{rgb}{0.95,0.95,1}
\newenvironment{quoteblock}{%
  \def\FrameCommand{%
    \hspace{1pt}%
    {\color{DarkBlue}\vrule width 2pt}%
    {\color{formalshade}\vrule width 4pt}%
    \colorbox{formalshade}%
  }%
  \MakeFramed{\advance\hsize-\width\FrameRestore}%
  \noindent\hspace{-4.55pt}%
  \begin{adjustwidth}{}{7pt}%
    \vspace{2pt}\vspace{2pt}%
  }
  {%
    \vspace{2pt}
  \end{adjustwidth}\endMakeFramed%
}

% 解答块样式
\definecolor{brownshade}{rgb}{0.99,0.97,0.93}
\newenvironment{solblock}{%
  \def\FrameCommand{%
    \hspace{1pt}%
    {\color{BurlyWood}\vrule width 2pt}%
    {\color{brownshade}\vrule width 4pt}%
    \colorbox{brownshade}%
  }%
  \MakeFramed{\advance\hsize-\width\FrameRestore}%
  \noindent\hspace{-4.55pt}%
  \begin{adjustwidth}{}{7pt}%
    \vspace{2pt}\vspace{2pt}%
  }
  {%
    \vspace{2pt}
  \end{adjustwidth}\endMakeFramed%
}

% 问题块样式
\definecolor{greenshade}{rgb}{0.90,0.99,0.91}
\newenvironment{quesblock}{%
  \def\FrameCommand{%
    \hspace{0.01pt}%
    {\color{Green}\vrule width 2pt}%
    {\color{greenshade}\vrule width 00.1pt}%
    \colorbox{greenshade}%
  }%
  \MakeFramed{\advance\hsize-\width\FrameRestore}%
  \noindent\hspace{-4.55pt}%
  \begin{adjustwidth}{}{7pt}%
    \vspace{2pt}\vspace{2pt}%
  }
  {%
    \vspace{2pt}
  \end{adjustwidth}\endMakeFramed%
}

% 标记块
\newtcolorbox{markerblock}[1][]{enhanced,
  before skip=2mm,after skip=3mm,
  boxrule=0.4pt,left=5mm,right=2mm,top=1mm,bottom=1mm,
  colback=yellow!20,
  colframe=yellow!40!black,
  sharp corners,rounded corners=southeast,arc is angular,arc=3mm,
  underlay={%
    \path[fill=tcbcolback!80!black] ([yshift=3mm]interior.south east)--++(-0.4,-0.1)--++(0.1,-0.2);
    \path[draw=tcbcolframe,shorten <=-0.05mm,shorten >=-0.05mm] ([yshift=3mm]interior.south east)--++(-0.4,-0.1)--++(0.1,-0.2);
    \path[fill=yellow!80!black,draw=none] (interior.south west) rectangle node[white]{\Huge\bfseries !} ([xshift=4mm]interior.north west);
  },
drop fuzzy shadow,#1}

% 提示块
\definecolor{tipscolor}{rgb}{0.77,0.72,0.65}
\newtcolorbox{tipsblock}[2][]
{enhanced,breakable,
  left=12pt,right=12pt,
  coltitle=white,
  colbacktitle=tipscolor,
  attach boxed title to top left={yshifttext=-1mm},
  boxed title style={skin=enhancedfirst jigsaw,arc=1mm,bottom=0mm,boxrule=0mm},
  boxrule=1pt,
  colback=OldLace,
  colframe=tipscolor,
  sharp corners=northwest,
  title=\vspace{3mm}\textbf{#2},
  arc=1mm,
#1}

% 定理类彩色块
\newenvironment{thmblock}[1][\textbf{Theorem}]{
\begin{tcolorbox}[title=\textbf{#1}, colback=red!5,colframe=red!75!black]}{
\end{tcolorbox}}

\newenvironment{defblock}[1][\textbf{Definition}]{
\begin{tcolorbox}[colback = Emerald!10, colframe = cyan!40!black, title = \textbf{#1}]}{
\end{tcolorbox}}

\newenvironment{lemmablock}[1][\textbf{Lemma}]{
\begin{tcolorbox}[title=\textbf{#1},colback=SeaGreen!10!CornflowerBlue!10,colframe=RoyalPurple!55!Aquamarine!100!]}{
\end{tcolorbox}}

\newenvironment{propblock}[1][\textbf{Proposition}]{
  \begin{tcolorbox}
  [title = \textbf{#1}, colback=Salmon!20, colframe=Salmon!90!Black]}{
\end{tcolorbox}}

\newenvironment{colblock}[1][\textbf{Collary}]{
\begin{tcolorbox}[colback=JungleGreen!10!Cerulean!15,colframe=CornflowerBlue!60!Black,title = \textbf{#1}]}{
\end{tcolorbox}}

% Mathematical operator definitions
\DeclareMathOperator{\im}{im}
\DeclareMathOperator{\coker}{coker}
\DeclareMathOperator{\ind}{ind}

\fancyhf{}
\fancypagestyle{plain}{
  \lhead{2025}
  \chead{\centering{Algebraic Curve}}
  \rhead{\thepage\ of \pageref{LastPage}}
  \lfoot{}
  \cfoot{}
\rfoot{}}
\pagestyle{plain}

%-------------------basic info-------------------

\title{\textbf{Algebraic Curve}}
% \author{Xinyu Xiang}
\date{Jun. 2025}

%-------------------document---------------------

\begin{document}
\maketitle

% Fields and 代数闭域:Z, Q, F_p
% AIM: Consider Bezout Theorem and RR Theorem
\section{Day I}

% \subsection{Very Very Briefly Introduction to Bezout Theorem}
%
% Consider the following polynomial equation:
% \begin{equation*}
%   \begin{cases}
%     y^{2} = x^{3} - x, \, \deg = 3\\
%     x = 1, \, \deg = 1.
%   \end{cases}
% \end{equation*}
% Solve the equation above, one could find that the solution is $ (1, 0)$, which could be interpreted as the intersection of the curve $ y = x^{3} - x$ and $ x = 1$.

\begin{definition}[Polynomial]
  The collection of polynomials would denoted by $ \mathbb{K}[x_1,\cdots ,x_{n}]$, whose elements are of the form
  \begin{equation*}
    f = \sum_{i_1,\cdots ,i_n} a_{i_1,\cdots ,i_n} x_1^{i_1}\cdots x_n^{i_n},
  \end{equation*}
  where $ a_{i_1,\cdots ,i_n} \in \mathbb{K}$, and $ i_1,\cdots ,i_n$ are non-negative integers.
\end{definition}

\begin{definition}[Algebraic Closed Field]
  If
\end{definition}

\begin{remark}
  Finite field is not algebraic closed: Consider $ f = (x - a_{1}) \cdots (x - a_{n}) + 1$ which has no zero point.
\end{remark}

\begin{definition}[Unique Factorization Domain (UFD)]

\end{definition}

\begin{proposition}
  \begin{enumerate}[(1)]
    \item   $ \mathbb{K}[x_1,\cdots ,x_{n}]$ is a commutative ring with unity called the polynomial ring in $ n$ variables over $ \mathbb{K}$.
    \item If $ R$ is UFD, then $ R[X]$ is a UFD, which means that every non-zero polynomial can be factored uniquely into irreducible polynomials, up to order and units.
  \end{enumerate}
\end{proposition}

From here on, we assume that $ \mathbb{K}$ is an algebraic closed field.
\begin{definition}[Affine Variety]
  An affine variety is a subset of $ \mathbb{K}^{n}$ defined by the vanishing of a set of polynomials, i.e., it is the solution set of a system of polynomial equations.

  Formally, given a set of polynomials $ f_1, \ldots, f_m \in \mathbb{K}[x_1,\ldots,x_n]$, the affine variety $ V(f_1, \ldots, f_m)$ is defined as:
  \begin{equation*}
    V(f_1, \ldots, f_m) = \{ (a_1, \ldots, a_n) \in \mathbb{K}^{n} ; f_i(a_1, \ldots, a_n) = 0 \text{ for all } i = 1, \ldots, m \}.
  \end{equation*}
\end{definition}

\begin{proposition}[Zariski Topology]
  Consider $ f, g \in \mathbb{K}[x,y]$
  \begin{enumerate}[(1)]
    \item $V(fg) = V(f) \cup V(g)$ ,
    \item $ V(f, g) = V(f) \cap V(g)$, $ V(f_{\lambda})_{\lambda \in \Lambda} = \bigcap_{\lambda \in \Lambda} V(f_{\lambda})$,
    \item $ V(0) = \mathbb{A}_{\mathbb{K}}^{2}$.
  \end{enumerate}
\end{proposition}

% 平面仿射曲线
\begin{definition}[Affine Curve]
  Consider $ f \in \mathbb{K}[x,y]$, $ V(f)$ denotes affine curve.
  \begin{enumerate}[(1)]
    \item $ \deg V(f) = \deg f$,
      \begin{enumerate}[(a)]
        \item $\deg = 1$: Line,
        \item $\deg = 2$: conic curve (non-degenerate),
      \end{enumerate}
    \item $ F = F_{1}^{n_1} F_2^{n_2} \cdots F_{m}^{n_{m}}$, where $ F_{i}$ irreducible.
  \end{enumerate}
\end{definition}

\begin{example}
  $( x + y )^{2}$ is irreducible, $ xy$ is reducible.
\end{example}
\begin{example}
  $ y^{2} - x^{3} + x$ is irreducible (left as exercise).
\end{example}

% 分式域
\begin{definition}[Field of Fractions]
  The field of fractions of a UFD $ R$ is the smallest field in which $ R$ can be embedded, denoted by $ K(R)$.
  It consists of elements of the form $ \frac{a}{b}$ where $ a, b \in R$ and $ b \neq 0 \in R$.

  Formally, if $ R$ is a UFD, then the field of fractions $ K(R)$ is defined as:
  \begin{equation*}
    Q_{\mathrm{uot}}(R) = \left\{ \frac{a}{b} \mid a, b \in R, b \neq 0 \right\},
  \end{equation*}
  which is indeed a field.
\end{definition}

\begin{lemma}
  Consider $ f \in \mathbb{K}[x,y]$ and $ \deg f > 0$, then
  \begin{enumerate}[(1)]
    \item $ V(f)$ has infinitely many points,
    \item $ \mathbb{A}^{2}_{\mathbb{K}} - V(f)$ has infinitely many points.
  \end{enumerate}
\end{lemma}

% 朱利安整环??
\begin{theorem}[Simple Bezout Theorem]
  If $ F, G \in \mathbb{K}[x,y] \subset \mathbb{K}(x)[y]$ has no common component, then $ V(F,G)$ is a finite set $\Leftrightarrow$ $ F=0$, $ G = 0$ have finite solutions in $ \mathbb{K}^{2}$.
\end{theorem}
\begin{proof}
  \begin{enumerate}[(1)]
    \item Assume there is an element $ \alpha$ such that $ F = \alpha F'$ and $ G = \alpha G'$, where we consider the ring $ \mathbb{K}(x)[y]$, then
      \begin{equation*}
        \begin{cases}
          a F = H F'\\
          b G = H G',
        \end{cases}
      \end{equation*}
      where $ a \in \mathbb{K}[x]$ and $ H \in \mathbb{K}[x,y]$.
    \item TBD
  \end{enumerate}
\end{proof}

\begin{theorem}
  Consider irreducible $ F, G \in \mathbb{K}[x,y]$, $ F | G \Leftrightarrow V(F) \subset V(G)$.
\end{theorem}
\begin{proof}
  \begin{enumerate}[(1)]
    \item If $ F | G$, then $ G = F H$ for some $ H \in \mathbb{K}[x,y]$, thus $ V(F) \subset V(G)$.
    \item If $ V(F) \subset V(G)$, by definition $ F | G$.
  \end{enumerate}
\end{proof}

\section{Day II: Intersection Number (1)}

\begin{definition}[Localized Ring]
  Consider $ \mathbb{K}[x,y]$ and a prime ideal $ P \subset R$, the localized ring $ \mathcal{O}_P$ is defined as:
  \begin{equation*}
    \mathcal{O}_P = \left\{ \frac{f}{g} ; f, g \in \mathbb{K}[x,y] , g(p) \neq 0 \right\},
  \end{equation*}
  the maximal ideal $ \mathfrak{m}_P$ is defined as:
  \begin{equation*}
    \mathfrak{m}_P = \left\{ \frac{f}{g} ; f,g \in \mathbb{K}[x,y] , g(p) \neq 0 , f(p) = 0\right\}.
  \end{equation*}
  which satisfies
  \begin{equation*}
    0 \rightarrow \mathfrak{m}_P \rightarrow \mathcal{O}_P \rightarrow \mathbb{K}.
  \end{equation*}
\end{definition}

\begin{definition}[Intersection Number]
  Consider $ F, G \in \mathbb{K}[x,y]$ irreducible, $ P = V(F) \cap V(G)$, then the intersection number $ I_P(F,G)$ is defined as:
  \begin{equation*}
    I_{P}(F,G) = \dim_{\mathbb{K}} \mathcal{O}_P / \left< F,G \right>,
  \end{equation*}
  where $\left< F,G \right>$ is the ideal generated by $ F$ and $ G$ in the localized ring $ \mathcal{O}_P$.
\end{definition}

\begin{proposition}
  \begin{enumerate}[(1)]
    \item $ I_{P}(F,G) \in \mathbb{N} \cup \left\{ \infty  \right\}$;
    \item $ P \in F \cap G \Leftrightarrow I_{P}(F,G) \ge 1$, $ I_{P}(F,G) = 1 \Leftrightarrow \left< F,G \right> = I_{P}$
    \item $ I_{P}(F,G) = I_{P}(G,F)$;
    \item $ I_{P}(F, G+FH) = I_{P}(F,G)$;
    \item $ I_{P}(FG, H) = I_{P}(F,H) + I_{P(G,H)};$
    \item $ I_{(0,0)}(x,y) = 1$,
  \end{enumerate}
\end{proposition}

\begin{example}
  Consider $ F = y - x^{2}$ and $ G = y$.
\end{example}
\begin{solution}
  \textbf{Use properties to compute}:
  \begin{equation*}
    \begin{aligned}
      I_{0}(y, y - x^{3}) = & I_{0}(y , - x^3) \\
      = & 2 I_0 (y,x) \\
      = & 2
    \end{aligned}
  \end{equation*}
  where we used the property (4) to reduce the degree of the polynomial for the given variable, and use the fact that $ I_{0}(x,y) = 1$ .
\end{solution}
The most important part is to use the property (4) to reduce the degree of the polynomial for the given variable.
\begin{example}
  Consider $ F = y^{2} - x^{3}$ and $ G = x^{2} - y^{3}$.
\end{example}
\begin{solution}
  \begin{equation*}
    \begin{aligned}
      I_{0}(y^{2} - x^{3}, x^{2} - y^{3}) = & I_{0}(y^{2} - x^{3} + x (x^{2} - y^{3}), y^{3} - x^{2}) \\
      = & I_0(y^{2} - x y^{3}, y^{3} - x^{2}) \\
      = & I_0(y^{2}, y^{3} - x^{2}) + I_0(1 - x y, y^{3} - x^{2}) \\
      = & 2 I_0(y, y^{3} - x^{2}) + 0\\
      = & 2 I_0(y, x^{2})\\
      = & 4 I_0(y,x) \\
      = & 4\\
    \end{aligned}
  \end{equation*}
  $ I_0(1-xy, y^{3} - x^{2})$ vanished since at $ (0,0)$, $ 1 - xy \neq 0$ and $ y^{3} - x^{2} = 0$.
\end{solution}

\begin{example}
  Consider $ F = y - x - x^{2}$ and $ G = y^{2} - x^{2} - 3 x^{2}y$.
\end{example}
\begin{solution}
  \begin{equation*}
    \begin{aligned}
      I_{0}(y-x- x^{2} , y^{2} - x^{2} - 3 x^{2}y) = & I_{0}(y - x - x^{2} , y^{2} - x^{2} - 3 x^{2}y - (x+y)(y - x - x^{2})) \\
      = & I_{0}(y - x - x^{2} , - 2 x^{2}y + x^{3}) \\
      = & I_0(y - x - x^{2}, x^{2} (x - 2y)) \\
      = & 2 I_0(y - x - x^{2}, x) + I_0(y - x - x^{2}, x - 2y) \\
      = & 3.
    \end{aligned}
  \end{equation*}
  Another way to compute is to use definition of intersection number, where we plug the equation $ y = x+x^{2}$ into the second equation, we have
  \begin{equation*}
    I_{0}(y - x - x^{2} , y^{2} - x^{2} - 3 x^{2}y) = \mathfrak{m}_0\left( (x+x^{2})^{2} - x^{2} - 3 x^{2}(x+x^{2}) \right) = \mathfrak{m}_0 \left( x^{3}(-1-2x) \right) = 0.
  \end{equation*}
\end{solution}

\label{LastPage}
\end{document}
