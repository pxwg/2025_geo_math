% !TeX program = pdflatex
\documentclass[10pt]{article}
\input{../preamble.tex}
\fancyhf{}
\fancypagestyle{plain}{
  \lhead{2025}
  \chead{\centering{Algebraic Curve}}
  \rhead{\thepage\ of \pageref{LastPage}}
  \lfoot{}
  \cfoot{}
\rfoot{}}
\pagestyle{plain}

%-------------------basic info-------------------

\title{\textbf{Algebraic Curve}}
% \author{Xinyu Xiang}
\date{Jun. 2025}

%-------------------document---------------------

\begin{document}
\maketitle

% Fields and 代数闭域:Z, Q, F_p
% AIM: Consider Bezout Theorem and RR Theorem
\section{Day I}

% \subsection{Very Very Briefly Introduction to Bezout Theorem}
%
% Consider the following polynomial equation:
% \begin{equation*}
%   \begin{cases}
%     y^{2} = x^{3} - x, \, \deg = 3\\
%     x = 1, \, \deg = 1.
%   \end{cases}
% \end{equation*}
% Solve the equation above, one could find that the solution is $ (1, 0)$, which could be interpreted as the intersection of the curve $ y = x^{3} - x$ and $ x = 1$.

\begin{definition}[Polynomial]
  The collection of polynomials would denoted by $ \mathbb{K}[x_1,\cdots ,x_{n}]$, whose elements are of the form
  \begin{equation*}
    f = \sum_{i_1,\cdots ,i_n} a_{i_1,\cdots ,i_n} x_1^{i_1}\cdots x_n^{i_n},
  \end{equation*}
  where $ a_{i_1,\cdots ,i_n} \in \mathbb{K}$, and $ i_1,\cdots ,i_n$ are non-negative integers.
\end{definition}

\begin{definition}[Algebraic Closed Field]
  If
\end{definition}

\begin{remark}
  Finite field is not algebraic closed: Consider $ f = (x - a_{1}) \cdots (x - a_{n}) + 1$ which has no zero point.
\end{remark}

\begin{definition}[Unique Factorization Domain (UFD)]

\end{definition}

\begin{proposition}
  \begin{enumerate}[(1)]
    \item   $ \mathbb{K}[x_1,\cdots ,x_{n}]$ is a commutative ring with unity called the polynomial ring in $ n$ variables over $ \mathbb{K}$.
    \item If $ R$ is UFD, then $ R[X]$ is a UFD, which means that every non-zero polynomial can be factored uniquely into irreducible polynomials, up to order and units.
  \end{enumerate}
\end{proposition}

From here on, we assume that $ \mathbb{K}$ is an algebraic closed field.
\begin{definition}[Affine Variety]
  An affine variety is a subset of $ \mathbb{K}^{n}$ defined by the vanishing of a set of polynomials, i.e., it is the solution set of a system of polynomial equations.

  Formally, given a set of polynomials $ f_1, \ldots, f_m \in \mathbb{K}[x_1,\ldots,x_n]$, the affine variety $ V(f_1, \ldots, f_m)$ is defined as:
  \begin{equation*}
    V(f_1, \ldots, f_m) = \{ (a_1, \ldots, a_n) \in \mathbb{K}^{n} ; f_i(a_1, \ldots, a_n) = 0 \text{ for all } i = 1, \ldots, m \}.
  \end{equation*}
\end{definition}

\begin{proposition}[Zariski Topology]
  Consider $ f, g \in \mathbb{K}[x,y]$
  \begin{enumerate}[(1)]
    \item $V(fg) = V(f) \cup V(g)$ ,
    \item $ V(f, g) = V(f) \cap V(g)$, $ V(f_{\lambda})_{\lambda \in \Lambda} = \bigcap_{\lambda \in \Lambda} V(f_{\lambda})$,
    \item $ V(0) = \mathbb{A}_{\mathbb{K}}^{2}$.
  \end{enumerate}
\end{proposition}

% 平面仿射曲线
\begin{definition}[Affine Curve]
  Consider $ f \in \mathbb{K}[x,y]$, $ V(f)$ denotes affine curve.
  \begin{enumerate}[(1)]
    \item $ \deg V(f) = \deg f$,
      \begin{enumerate}[(a)]
        \item $\deg = 1$: Line,
        \item $\deg = 2$: conic curve (non-degenerate),
      \end{enumerate}
    \item $ F = F_{1}^{n_1} F_2^{n_2} \cdots F_{m}^{n_{m}}$, where $ F_{i}$ irreducible.
  \end{enumerate}
\end{definition}

\begin{example}
  $( x + y )^{2}$ is irreducible, $ xy$ is reducible.
\end{example}
\begin{example}
  $ y^{2} - x^{3} + x$ is irreducible (left as exercise).
\end{example}

% 分式域
\begin{definition}[Field of Fractions]
  The field of fractions of a UFD $ R$ is the smallest field in which $ R$ can be embedded, denoted by $ K(R)$.
  It consists of elements of the form $ \frac{a}{b}$ where $ a, b \in R$ and $ b \neq 0 \in R$.

  Formally, if $ R$ is a UFD, then the field of fractions $ K(R)$ is defined as:
  \begin{equation*}
    Q_{\mathrm{uot}}(R) = \left\{ \frac{a}{b} \mid a, b \in R, b \neq 0 \right\},
  \end{equation*}
  which is indeed a field.
\end{definition}

\begin{lemma}
  Consider $ f \in \mathbb{K}[x,y]$ and $ \deg f > 0$, then
  \begin{enumerate}[(1)]
    \item $ V(f)$ has infinitely many points,
    \item $ \mathbb{A}^{2}_{\mathbb{K}} - V(f)$ has infinitely many points.
  \end{enumerate}
\end{lemma}

% 朱利安整环??
\begin{theorem}[Simple Bezout Theorem]
  If $ F, G \in \mathbb{K}[x,y] \subset \mathbb{K}(x)[y]$ has no common component, then $ V(F,G)$ is a finite set $\Leftrightarrow$ $ F=0$, $ G = 0$ have finite solutions in $ \mathbb{K}^{2}$.
\end{theorem}
\begin{proof}
  \begin{enumerate}[(1)]
    \item Assume there is an element $ \alpha$ such that $ F = \alpha F'$ and $ G = \alpha G'$, where we consider the ring $ \mathbb{K}(x)[y]$, then
      \begin{equation*}
        \begin{cases}
          a F = H F'\\
          b G = H G',
        \end{cases}
      \end{equation*}
      where $ a \in \mathbb{K}[x]$ and $ H \in \mathbb{K}[x,y]$.
    \item TBD
  \end{enumerate}
\end{proof}

\begin{theorem}
  Consider irreducible $ F, G \in \mathbb{K}[x,y]$, $ F | G \Leftrightarrow V(F) \subset V(G)$.
\end{theorem}
\begin{proof}
  \begin{enumerate}[(1)]
    \item If $ F | G$, then $ G = F H$ for some $ H \in \mathbb{K}[x,y]$, thus $ V(F) \subset V(G)$.
    \item If $ V(F) \subset V(G)$, by definition $ F | G$.
  \end{enumerate}
\end{proof}

\section{Day II: Intersection Number (1)}

\begin{definition}[Localized Ring]
  Consider $ \mathbb{K}[x,y]$ and a prime ideal $ P \subset R$, the localized ring $ \mathcal{O}_P$ is defined as:
  \begin{equation*}
    \mathcal{O}_P = \left\{ \frac{f}{g} ; f, g \in \mathbb{K}[x,y] , g(p) \neq 0 \right\},
  \end{equation*}
  the maximal ideal $ \mathfrak{m}_P$ is defined as:
  \begin{equation*}
    \mathfrak{m}_P = \left\{ \frac{f}{g} ; f,g \in \mathbb{K}[x,y] , g(p) \neq 0 , f(p) = 0\right\}.
  \end{equation*}
  which satisfies
  \begin{equation*}
    0 \rightarrow \mathfrak{m}_P \rightarrow \mathcal{O}_P \rightarrow \mathbb{K}.
  \end{equation*}
\end{definition}

\label{LastPage}
\end{document}
