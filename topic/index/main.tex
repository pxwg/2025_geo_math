% !TeX program = xelatex
\documentclass[10pt]{article}
\input{../preamble.tex}
\fancyhf{}
\fancypagestyle{plain}{
  \lhead{2025}
  \chead{\centering{Introduction to Index Theory}}
  \rhead{\thepage\ of \pageref{LastPage}}
  \lfoot{}
  \cfoot{}
\rfoot{}}
\pagestyle{plain}

%-------------------fonts-------------------
\usepackage[sc]{mathpazo}
\usepackage{courier}
\usepackage[utf8]{inputenc}
\usepackage[T1]{fontenc}
\usepackage{microtype}
\RequirePackage[font=small,format=plain,labelfont=bf,textfont=it]{caption}

%-------------------basic info-------------------

\title{\textbf{Introduction to Index Theory}}
\author{Xinyu Xiang}
\date{Jul. 2025}

%-------------------document---------------------

\begin{document}
\maketitle

\section{Motivation}

Consider $ T \in \mathcal{B}(\mathcal{H})$ which is a bounded operator in Hilbert space. The decomposition $ H = \ker(T) \oplus H_0$ and $ H = \in(T) \oplus H_1$ leads to an isomorphism (which is finite dimensional)
\begin{equation*}
  T |_{H_0 } : H_0 \rightarrow \im(T).
\end{equation*}
where the index of $ T$ is
\begin{equation*}
  \mathrm{ind}(T) = \dim \ker (T) - \dim \coker (T).
\end{equation*}
If index of $ T$ is $ 0 $, there would be no obstruction to extend an operator to an isomorphic one, i.e. index captured the obstruction to extend to an isomorphism.

Consider an continuous transformation family of $ T$ which would be written as $ \left( T_{t} \right)_{t \in [0,1]} $. Which could be written as
\begin{equation*}
  \mathrm{ind}(T_{1}) = \mathrm{ind}(T_0).
\end{equation*}
Thus, one could expect the index of an operator is homotopy invariant.

\begin{example}[The Topology of Harday Space]
  Consider Harday space
  \begin{equation*}
    H^{2}(\mathbb{S}^{1}) = \left\{ \sum_{n \in \mathbb{N}} a_{n} z^{n}, \sum_{n \in \mathbb{N}} \left| a_n \right|^{2} < \infty \right\}
  \end{equation*}
  There would be an projection operator $ p : L^{2}(\mathbb{S}^{1}) \rightarrow H^{2}(\mathbb{S}^{1})$, $ f \in C(\mathbb{S}^{1})$ and there is a multplication operator $ M_{f} : L^{2}(\mathbb{S}^{1}) \rightarrow L^{2}(\mathbb{S}^{1})$.
  Finally, one could consider the topological operator $ p M_{f} p : H^{2}(\mathbb{S}^{1}) \rightarrow H^{2}(\mathbb{S}^{1})$

  Now consider $ f(z) = z$ and $ f(z) = z^{-1}$, the shift operator $ T_{f} : H^{2}(\mathbb{S}^{1}) \rightarrow H^{2}(\mathbb{S}^{1})$ is defined as
  \begin{equation*}
    (a_0, a_1, a_2, \ldots) \mapsto (0, a_0, a_1, \ldots), \quad \text{and } (a_0, a_1, a_2, \ldots) \mapsto (a_1, a_2, \ldots).
  \end{equation*}
  \begin{theorem}[Noether, Fuly 1935]
    Suppose $ f \neq 0$ for all $ z \in \mathbb{S}^{2}$ (i.e. $ f$ is a loop on $ \mathbb{C}- {0}$), we have $ \mathrm{ind}(T_{f}) = - \mathrm{winding}(f)$.
  \end{theorem}
  \begin{proof}
    The index is depends on on homotopy class of $ f: \mathbb{S}^{1} \rightarrow \mathbb{C}-\left\{ 0 \right\}$ which could be characterized by $ \pi_1(\mathbb{C}-\left\{ 0 \right\}) = \mathbb{Z}$. So that $[f] = [z^{k}]$ where $ k$ is the dimension of cokernel i.e. the index of $ f$, which is associated with the winding number of $ f$, thus one have
    \begin{equation*}
      \mathrm{ind}(T_{f}) = - \mathrm{winding}(f)
    \end{equation*}
    which proved the theorem.
  \end{proof}
\end{example}

\section{A-S Index Theorem}

Consider Elliptic differential operator $ \mathcal{D} = a_{\alpha}(x) \partial^{\alpha} \xrightarrow \sigma(x, \xi) \equiv a_{\alpha} \xi^{\alpha}$ which could be defined by Fourier transformation,
where $ \xi^{\alpha}$ for $ \xi = (\xi_{1},\cdots ,\xi_{n})$ could be defined  $ \xi^{\alpha} = (\xi_{1}^{\alpha}, \cdots ,\xi_{n}^{\alpha})$ and $ a_{\alpha}$ is matrix take value in $ C^{\infty }_{c}(\mathbb{R}^{n})$.
Elliptic means $ \sigma \in GL_{n}(\mathbb{C})$.

\label{LastPage}
\end{document}
