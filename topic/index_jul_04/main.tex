% !TeX program = xelatex
% ltex: enabled=false
% TODO: fix typo
\documentclass[10pt]{article}
%% ======================================================================
%% LaTeX Preamble for Physics Note, Experiment Report, and Research in
%% CJK and English
%% ======================================================================

%% ----------------------------------------------------------------------
%% 1. 文档基础设置与语言支持
%% ----------------------------------------------------------------------
\usepackage{geometry}
\geometry{a4paper,centering,scale=0.75} % 页面设置

% 中文支持 - 适配 macOS
% \usepackage[UTF8]{ctex}
% \AtBeginDocument{\small} % 全局缩小字体
% \setCJKmainfont[
%   Script=CJK,
%   BoldFont={Heiti SC},
%   ItalicFont={Kaiti SC}
% ]{Songti SC}

%% ----------------------------------------------------------------------
%% 2. 数学支持包
%% ----------------------------------------------------------------------
\usepackage{amsmath,amssymb,amsthm} % 数学基础支持
\usepackage{mathtools} % amsmath 扩展
\usepackage{physics} % 物理公式简化
\usepackage{mathrsfs} % 数学花体字
\usepackage{bbm} % 黑板粗体
\usepackage{cancel} % 公式划除线
\numberwithin{equation}{section} % 按节编号公式
\usepackage[sc]{mathpazo}
\usepackage{courier}
\usepackage{inputenc}
\usepackage[T1]{fontenc}
\usepackage{microtype}
\RequirePackage[font=small,format=plain,labelfont=bf,textfont=it]{caption}

%% ----------------------------------------------------------------------
%% 3. 图形与绘图支持
%% ----------------------------------------------------------------------
\usepackage{graphicx} % 图片支持
\usepackage[export]{adjustbox} % 图片调整
\usepackage{float} % 控制浮动体
\usepackage{pgfplots} % 绘图支持
\pgfplotsset{compat=newest} % 使用最新版本特性
% \usepackage[compat=1.1.0,warn luatex=false]{tikz-feynman} % Feynman 图
\usepackage{quiver} % 交换图

%% ----------------------------------------------------------------------
%% 4. 表格支持
%% ----------------------------------------------------------------------
\usepackage{tabularx} % 增强表格
\usepackage{xltabular} % 长表格支持

%% ----------------------------------------------------------------------
%% 5. 交叉引用与超链接
%% ----------------------------------------------------------------------
% hyperref 通常放在最后加载以避免冲突
\usepackage[bookmarks=true, colorlinks=true, linkcolor=teal,
citecolor=blue, urlcolor=magenta, hidelinks]{hyperref}

%% ----------------------------------------------------------------------
%% 6. 外观与装饰
%% ----------------------------------------------------------------------
\usepackage{fancyhdr} % 页眉页脚
\usepackage[dvipsnames,svgnames]{xcolor} % 扩展颜色支持
\usepackage{framed} % 框架效果
\usepackage{tcolorbox} % 彩色文本框
\tcbuselibrary{most} % tcolorbox 扩展库
\usepackage[strict]{changepage} % 提供 adjustwidth 环境
\usepackage{scalerel} % 缩放支持

%% ----------------------------------------------------------------------
%% 7. 引用与杂项支持
%% ----------------------------------------------------------------------
\usepackage{enumerate} % 列表环境
\usepackage{stackrel} % 符号堆叠
\usepackage{import,xifthen,pdfpages} % 文档导入与条件判断

% 条件加载透明效果包 (仅在 PDF 模式下)
\usepackage{ifpdf}
\ifpdf
\usepackage{transparent}
\else
% 非 PDF 模式下使用替代方案或提供警告
\newcommand{\transparent}[1]{}
\typeout{警告:transparent 包功能在非 PDF 模式下不可用}
\fi

%% ----------------------------------------------------------------------
%% 8. 定理环境设置
%% ----------------------------------------------------------------------
% English theorem environments
\newtheorem{theorem}{Theorem}[section]
\newtheorem{lemma}[theorem]{Lemma}
\newtheorem{proposition}[theorem]{Proposition}
\newtheorem{corollary}[theorem]{Corollary}
\newtheorem{definition}{Definition}[section]
\newtheorem{remark}{Remark}[section]
\newtheorem{example}{Example}[section]
\newtheorem{construction}{Construction}[section]
\newenvironment{observation}{
\begin{proof}[Observation]}{
\end{proof}}
\newenvironment{solution}{
\begin{proof}[Solution]}{
\end{proof}}

%% ----------------------------------------------------------------------
%% 9. 自定义框架和环境
%% ----------------------------------------------------------------------
% 引用块样式
\definecolor{formalshade}{rgb}{0.95,0.95,1}
\newenvironment{quoteblock}{%
  \def\FrameCommand{%
    \hspace{1pt}%
    {\color{DarkBlue}\vrule width 2pt}%
    {\color{formalshade}\vrule width 4pt}%
    \colorbox{formalshade}%
  }%
  \MakeFramed{\advance\hsize-\width\FrameRestore}%
  \noindent\hspace{-4.55pt}%
  \begin{adjustwidth}{}{7pt}%
    \vspace{2pt}\vspace{2pt}%
  }
  {%
    \vspace{2pt}
  \end{adjustwidth}\endMakeFramed%
}

% 解答块样式
\definecolor{brownshade}{rgb}{0.99,0.97,0.93}
\newenvironment{solblock}{%
  \def\FrameCommand{%
    \hspace{1pt}%
    {\color{BurlyWood}\vrule width 2pt}%
    {\color{brownshade}\vrule width 4pt}%
    \colorbox{brownshade}%
  }%
  \MakeFramed{\advance\hsize-\width\FrameRestore}%
  \noindent\hspace{-4.55pt}%
  \begin{adjustwidth}{}{7pt}%
    \vspace{2pt}\vspace{2pt}%
  }
  {%
    \vspace{2pt}
  \end{adjustwidth}\endMakeFramed%
}

% 问题块样式
\definecolor{greenshade}{rgb}{0.90,0.99,0.91}
\newenvironment{quesblock}{%
  \def\FrameCommand{%
    \hspace{0.01pt}%
    {\color{Green}\vrule width 2pt}%
    {\color{greenshade}\vrule width 00.1pt}%
    \colorbox{greenshade}%
  }%
  \MakeFramed{\advance\hsize-\width\FrameRestore}%
  \noindent\hspace{-4.55pt}%
  \begin{adjustwidth}{}{7pt}%
    \vspace{2pt}\vspace{2pt}%
  }
  {%
    \vspace{2pt}
  \end{adjustwidth}\endMakeFramed%
}

% 标记块
\newtcolorbox{markerblock}[1][]{enhanced,
  before skip=2mm,after skip=3mm,
  boxrule=0.4pt,left=5mm,right=2mm,top=1mm,bottom=1mm,
  colback=yellow!20,
  colframe=yellow!40!black,
  sharp corners,rounded corners=southeast,arc is angular,arc=3mm,
  underlay={%
    \path[fill=tcbcolback!80!black] ([yshift=3mm]interior.south east)--++(-0.4,-0.1)--++(0.1,-0.2);
    \path[draw=tcbcolframe,shorten <=-0.05mm,shorten >=-0.05mm] ([yshift=3mm]interior.south east)--++(-0.4,-0.1)--++(0.1,-0.2);
    \path[fill=yellow!80!black,draw=none] (interior.south west) rectangle node[white]{\Huge\bfseries !} ([xshift=4mm]interior.north west);
  },
drop fuzzy shadow,#1}

% 提示块
\definecolor{tipscolor}{rgb}{0.77,0.72,0.65}
\newtcolorbox{tipsblock}[2][]
{enhanced,breakable,
  left=12pt,right=12pt,
  coltitle=white,
  colbacktitle=tipscolor,
  attach boxed title to top left={yshifttext=-1mm},
  boxed title style={skin=enhancedfirst jigsaw,arc=1mm,bottom=0mm,boxrule=0mm},
  boxrule=1pt,
  colback=OldLace,
  colframe=tipscolor,
  sharp corners=northwest,
  title=\vspace{3mm}\textbf{#2},
  arc=1mm,
#1}

% 定理类彩色块
\newenvironment{thmblock}[1][\textbf{Theorem}]{
\begin{tcolorbox}[title=\textbf{#1}, colback=red!5,colframe=red!75!black]}{
\end{tcolorbox}}

\newenvironment{defblock}[1][\textbf{Definition}]{
\begin{tcolorbox}[colback = Emerald!10, colframe = cyan!40!black, title = \textbf{#1}]}{
\end{tcolorbox}}

\newenvironment{lemmablock}[1][\textbf{Lemma}]{
\begin{tcolorbox}[title=\textbf{#1},colback=SeaGreen!10!CornflowerBlue!10,colframe=RoyalPurple!55!Aquamarine!100!]}{
\end{tcolorbox}}

\newenvironment{propblock}[1][\textbf{Proposition}]{
  \begin{tcolorbox}
  [title = \textbf{#1}, colback=Salmon!20, colframe=Salmon!90!Black]}{
\end{tcolorbox}}

\newenvironment{colblock}[1][\textbf{Collary}]{
\begin{tcolorbox}[colback=JungleGreen!10!Cerulean!15,colframe=CornflowerBlue!60!Black,title = \textbf{#1}]}{
\end{tcolorbox}}

% Mathematical operator definitions
\DeclareMathOperator{\im}{im}
\DeclareMathOperator{\coker}{coker}
\DeclareMathOperator{\ind}{ind}

\fancyhf{}
\fancypagestyle{plain}{
  \lhead{2025}
  \chead{\centering{Introduction to Index Theory}}
  \rhead{\thepage\ of \pageref{LastPage}}
  \lfoot{}
  \cfoot{}
\rfoot{}}
\pagestyle{plain}

%-------------------fonts-------------------
\usepackage[sc]{mathpazo}
\usepackage{slashed}
\usepackage{courier}
% \usepackage[utf8]{inputenc}
\usepackage[T1]{fontenc}
\usepackage{microtype}
\RequirePackage[font=small,format=plain,labelfont=bf,textfont=it]{caption}

%-------------------basic info-------------------

\title{\textbf{Introduction to Index Theory}}
\author{Xinyu Xiang}
\date{Jul. 2025}

%-------------------document---------------------

\begin{document}
\maketitle

\section{Motivation}

Consider $ T \in \mathcal{B}(\mathcal{H})$ which is a bounded operator in Hilbert space. The decomposition $ H = \ker(T) \oplus H_0$ and $ H = \in(T) \oplus H_1$ leads to an isomorphism (which is finite dimensional)
\begin{equation*}
  T |_{H_0 } : H_0 \rightarrow \im(T).
\end{equation*}
where the index of $ T$ is
\begin{equation*}
  \mathrm{ind}(T) = \dim \ker (T) - \dim \coker (T).
\end{equation*}
If index of $ T$ is $ 0 $, there would be no obstruction to extend an operator to an isomorphic one, i.e. index captured the obstruction to extend to an isomorphism.

Consider an continuous transformation family of $ T$ which would be written as $ \left( T_{t} \right)_{t \in [0,1]} $. Which could be written as
\begin{equation*}
  \mathrm{ind}(T_{1}) = \mathrm{ind}(T_0).
\end{equation*}
Thus, one could expect the index of an operator is homotopy invariant.

\begin{example}[The Topology of Harday Space]
  Consider Harday space
  \begin{equation*}
    H^{2}(\mathbb{S}^{1}) = \left\{ \sum_{n \in \mathbb{N}} a_{n} z^{n}, \sum_{n \in \mathbb{N}} \left| a_n \right|^{2} < \infty \right\}
  \end{equation*}
  There would be an projection operator $ p : L^{2}(\mathbb{S}^{1}) \rightarrow H^{2}(\mathbb{S}^{1})$, $ f \in C(\mathbb{S}^{1})$ and there is a multplication operator $ M_{f} : L^{2}(\mathbb{S}^{1}) \rightarrow L^{2}(\mathbb{S}^{1})$.
  Finally, one could consider the topological operator $ p M_{f} p : H^{2}(\mathbb{S}^{1}) \rightarrow H^{2}(\mathbb{S}^{1})$

  Now consider $ f(z) = z$ and $ f(z) = z^{-1}$, the shift operator $ T_{f} : H^{2}(\mathbb{S}^{1}) \rightarrow H^{2}(\mathbb{S}^{1})$ is defined as
  \begin{equation*}
    (a_0, a_1, a_2, \ldots) \mapsto (0, a_0, a_1, \ldots), \quad \text{and } (a_0, a_1, a_2, \ldots) \mapsto (a_1, a_2, \ldots).
  \end{equation*}
  \begin{theorem}[Noether, Fuly 1935]
    Suppose $ f \neq 0$ for all $ z \in \mathbb{S}^{2}$ (i.e. $ f$ is a loop on $ \mathbb{C}- {0}$), we have $ \mathrm{ind}(T_{f}) = - \mathrm{winding}(f)$.
  \end{theorem}
  \begin{proof}
    The index is depends on on homotopy class of $ f: \mathbb{S}^{1} \rightarrow \mathbb{C}-\left\{ 0 \right\}$ which could be characterized by $ \pi_1(\mathbb{C}-\left\{ 0 \right\}) = \mathbb{Z}$. So that $[f] = [z^{k}]$ where $ k$ is the dimension of cokernel i.e. the index of $ f$, which is associated with the winding number of $ f$, thus one have
    \begin{equation*}
      \mathrm{ind}(T_{f}) = - \mathrm{winding}(f)
    \end{equation*}
    which proved the theorem.
  \end{proof}
\end{example}

\section{A-S Index Theorem}

Consider Elliptic differential operator $ \mathcal{D} = a_{\alpha}(x) \partial^{\alpha} \xrightarrow \sigma(x, \xi) \equiv a_{\alpha} \xi^{\alpha}$ which could be defined by Fourier transformation,
where $ \xi^{\alpha}$ for $ \xi = (\xi_{1},\cdots ,\xi_{n})$ could be defined  $ \xi^{\alpha} = (\xi_{1}^{\alpha}, \cdots ,\xi_{n}^{\alpha})$ and $ a_{\alpha}$ is matrix take value in $ C^{\infty }_{c}(\mathbb{R}^{n})$.
Elliptic means $ \sigma \in GL_{n}(\mathbb{C})$.

$ \slashed{\mathcal{D}} = \gamma_{i} \partial_{i}$ is Dirac operator where
\begin{equation*}
  \begin{cases}
    \gamma_{i}^{2} = 1, & \text{for } i = 1, \cdots , n,\\
    \gamma_{i} \gamma_{j} + \gamma_{j} \gamma_{i} = 0, & \text{for } i \neq j.
  \end{cases}
\end{equation*}
Now we want to define the Dirac operator $ \slashed{\mathcal{D}}$ on a manifold $ M$ with a Riemannian metric $ g$.
$\slashed{\mathcal{D}} ^{2} = \Delta$ is always well defined on $ M$.
If one want to extend the definition of Dirac operator, one need to introduce a spin structure on $ M$ i.e. taking an universal cover over $ SO(n) $ which is the automorphism group of $ \Delta$.
The result would lead to spin group $ Spin(n)$ which is a double cover of $ SO(n)$, and the spinor bundle $ S$ is defined as the associated vector bundle of $ Spin(n)$.
A manifold with a spin structure is called a spin manifold, where is the Dirac operator lives in.
\begin{theorem}
  Let $ \hat{A}$ be the A-S characteristic class of a spin manifold $ M$ with dimension $ n$, we have
  \begin{equation*}
    \int _{M} \hat{A}(M) \in \mathbb{Z}.
  \end{equation*}
\end{theorem}
Here, $ \hat{A}(M)$ a characteristic class of a spin manifold $ M$ with dimension $ n$ and curvature $ R$, which is defined as
\begin{equation*}
  \hat{A}(M) = \left( \frac{R}{\sinh R} \right)^{1 / 2} \in H^{\bullet}(M).
\end{equation*}

\subsection{Interlude: Euler Characteristic and Ball}

It is well known that $ \chi(M) = 2 - 2g$ for a closed orientable surface $ M$ with genus $ g$. Use this consideration to $ M = \mathbb{S}^{2}$, one could consider the vector field over $ \mathbb{S}^{2}$, where $ \chi(\mathbb{S}^{2}) = 2$.

% TODO: make it more precise
Consider the vector filed $ V : M \rightarrow T M$, which could be interpreted as a section of the tangent bundle $ TM$, which could be also be written alternatively as the diagnal of embadding $ M \rightarrow M \times M$ thus one have
\begin{equation*}
  \chi(\mathbb{S}^{2}) = \# (M, M),
\end{equation*}
which implies that the number of zeroes of a vector field on $ M$ is equal to the Euler characteristic of $ M$.

\subsection{Thom Class}

The characteristic class could be viewed as the degree of 'non linearly independence' of the frame of the vector bundle $ E \rightarrow M$.

Generalize this idea to the complex vector bundle $ E \rightarrow M$ of rank $ k$, then the obstruction of the existence of linear independent frame is equivalent to the existence of characteristic class of the vector bundle $ E$ which is $ c_{i} \in H^{2i} (M)$.
And the Pontryagin class would become $ p_{i}(E) = c_{2i} (E \otimes \mathbb{C})$.

\begin{theorem}[Atiyah-Singer Index Theorem]
  Let $ M$ be a closed spin manifold of dimension $ n$, and $ \mathcal{D}$ be an elliptic operator on $ M$ with index $ \mathrm{ind}(\mathcal{D})$. Then
  \begin{equation*}
    \mathrm{ind}(\mathcal{D}) = \int_{TM} \operatorname{Todd}(TM \otimes  \mathbb{C}) \operatorname{ch}(\sigma_{D}),
  \end{equation*}
  where $ E$ is the vector bundle associated with the elliptic operator $ \mathcal{D}$, $ \sigma_{\mathcal{D}} \in K(TM)$.
\end{theorem}

\subsection{\texorpdfstring{Relation to $K$ Theory}{Relation to K Theory}}

\section{Application}

Gauss-Bonnet theorem $ D = \mathrm{d} + \mathrm{d} ^{*}$; Riemann-Roch theorem: $ D = \bar{\partial} + \bar{\partial}^{*}$

\subsection{Signature Theorem}

Consider the intersection number $\# (V,W) = \int _{M} P_{V}\wedge P_{W}$ where $ P_{V}$ is the Poincaré dual of the vector field $ V$. Let $ M$ be an closed oriented manifold, then the intersection number $ H^{2k}(M) \times H^{2k}(M) \rightarrow \mathbb{R}$ could be written as
\begin{equation*}
  \left( \alpha,\beta \right) \mapsto \int_{M} \alpha \wedge \beta,
\end{equation*}
which is a quadratic form $ \eta$ on $ H^{2k}(M)$, and then one could define the signature of $ M$ as the signature of the quadratic form $ \eta$ on $ H^{2k}(M)$, which is defined as
\begin{equation*}
  \sigma(M) = \text{Signature}(M) = \left( \text{\# > 0 Eigenvalues} \right) - \left( \text{\# < 0 Eigenvalues} \right).
\end{equation*}

\begin{theorem}[Signature Theorem]
  Let $ M$ be a closed oriented manifold of dimension $ n$, then
  \begin{equation*}
    \sigma(M) = \int _{M} L(M).
  \end{equation*}
\end{theorem}
\begin{proof}
  We have an algebra homomorphism: $ \sigma: \Omega^{SO}_{4 k} \otimes \mathbb{Q} \rightarrow \mathbb{Q}$ where $ \Omega_{4k}^{SO}$ is the cobordism group of oriented manifolds of dimension $ 4k$. However, we have an isomorphism
  \begin{equation*}
    \Omega^{SO}_{4k} \otimes \mathbb{Q} \cong \mathbb{Q}[\mathbb{P}^{2}(\mathbb{C}), \mathbb{P}^{2}(\mathbb{C}),\cdots ],
  \end{equation*}
  which is the polynomial algebra generated by the oriented cobordism classes $ \mathbb{P}^{2j}[\mathbb{C}]$, it's enough to verify
  \begin{equation*}
    \sigma (\mathbb {P} ^{2i})=1=\langle L_{i}(p_{1}(\mathbb {P} ^{2i}),\ldots ,p_{n}(\mathbb {P} ^{2i})),[\mathbb {P} ^{2i}]\rangle.
  \end{equation*}
\end{proof}

\section{NCG Approach}

\label{LastPage}
\end{document}
