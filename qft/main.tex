% !TeX program = pdflatex
% ltex: enabled=false
\documentclass[10pt]{article}
%% ======================================================================
%% LaTeX Preamble for Physics Note, Experiment Report, and Research in
%% CJK and English
%% ======================================================================

%% ----------------------------------------------------------------------
%% 1. 文档基础设置与语言支持
%% ----------------------------------------------------------------------
\usepackage{geometry}
\geometry{a4paper,centering,scale=0.75} % 页面设置

% 中文支持 - 适配 macOS
% \usepackage[UTF8]{ctex}
% \AtBeginDocument{\small} % 全局缩小字体
% \setCJKmainfont[
%   Script=CJK,
%   BoldFont={Heiti SC},
%   ItalicFont={Kaiti SC}
% ]{Songti SC}

%% ----------------------------------------------------------------------
%% 2. 数学支持包
%% ----------------------------------------------------------------------
\usepackage{amsmath,amssymb,amsthm} % 数学基础支持
\usepackage{mathtools} % amsmath 扩展
\usepackage{physics} % 物理公式简化
\usepackage{mathrsfs} % 数学花体字
\usepackage{bbm} % 黑板粗体
\usepackage{cancel} % 公式划除线
\numberwithin{equation}{section} % 按节编号公式
\usepackage[sc]{mathpazo}
\usepackage{courier}
\usepackage{inputenc}
\usepackage[T1]{fontenc}
\usepackage{microtype}
\RequirePackage[font=small,format=plain,labelfont=bf,textfont=it]{caption}

%% ----------------------------------------------------------------------
%% 3. 图形与绘图支持
%% ----------------------------------------------------------------------
\usepackage{graphicx} % 图片支持
\usepackage[export]{adjustbox} % 图片调整
\usepackage{float} % 控制浮动体
\usepackage{pgfplots} % 绘图支持
\pgfplotsset{compat=newest} % 使用最新版本特性
% \usepackage[compat=1.1.0,warn luatex=false]{tikz-feynman} % Feynman 图
\usepackage{quiver} % 交换图

%% ----------------------------------------------------------------------
%% 4. 表格支持
%% ----------------------------------------------------------------------
\usepackage{tabularx} % 增强表格
\usepackage{xltabular} % 长表格支持

%% ----------------------------------------------------------------------
%% 5. 交叉引用与超链接
%% ----------------------------------------------------------------------
% hyperref 通常放在最后加载以避免冲突
\usepackage[bookmarks=true, colorlinks=true, linkcolor=teal,
citecolor=blue, urlcolor=magenta, hidelinks]{hyperref}

%% ----------------------------------------------------------------------
%% 6. 外观与装饰
%% ----------------------------------------------------------------------
\usepackage{fancyhdr} % 页眉页脚
\usepackage[dvipsnames,svgnames]{xcolor} % 扩展颜色支持
\usepackage{framed} % 框架效果
\usepackage{tcolorbox} % 彩色文本框
\tcbuselibrary{most} % tcolorbox 扩展库
\usepackage[strict]{changepage} % 提供 adjustwidth 环境
\usepackage{scalerel} % 缩放支持

%% ----------------------------------------------------------------------
%% 7. 引用与杂项支持
%% ----------------------------------------------------------------------
\usepackage{enumerate} % 列表环境
\usepackage{stackrel} % 符号堆叠
\usepackage{import,xifthen,pdfpages} % 文档导入与条件判断

% 条件加载透明效果包 (仅在 PDF 模式下)
\usepackage{ifpdf}
\ifpdf
\usepackage{transparent}
\else
% 非 PDF 模式下使用替代方案或提供警告
\newcommand{\transparent}[1]{}
\typeout{警告:transparent 包功能在非 PDF 模式下不可用}
\fi

%% ----------------------------------------------------------------------
%% 8. 定理环境设置
%% ----------------------------------------------------------------------
% English theorem environments
\newtheorem{theorem}{Theorem}[section]
\newtheorem{lemma}[theorem]{Lemma}
\newtheorem{proposition}[theorem]{Proposition}
\newtheorem{corollary}[theorem]{Corollary}
\newtheorem{definition}{Definition}[section]
\newtheorem{remark}{Remark}[section]
\newtheorem{example}{Example}[section]
\newtheorem{construction}{Construction}[section]
\newenvironment{observation}{
\begin{proof}[Observation]}{
\end{proof}}
\newenvironment{solution}{
\begin{proof}[Solution]}{
\end{proof}}

%% ----------------------------------------------------------------------
%% 9. 自定义框架和环境
%% ----------------------------------------------------------------------
% 引用块样式
\definecolor{formalshade}{rgb}{0.95,0.95,1}
\newenvironment{quoteblock}{%
  \def\FrameCommand{%
    \hspace{1pt}%
    {\color{DarkBlue}\vrule width 2pt}%
    {\color{formalshade}\vrule width 4pt}%
    \colorbox{formalshade}%
  }%
  \MakeFramed{\advance\hsize-\width\FrameRestore}%
  \noindent\hspace{-4.55pt}%
  \begin{adjustwidth}{}{7pt}%
    \vspace{2pt}\vspace{2pt}%
  }
  {%
    \vspace{2pt}
  \end{adjustwidth}\endMakeFramed%
}

% 解答块样式
\definecolor{brownshade}{rgb}{0.99,0.97,0.93}
\newenvironment{solblock}{%
  \def\FrameCommand{%
    \hspace{1pt}%
    {\color{BurlyWood}\vrule width 2pt}%
    {\color{brownshade}\vrule width 4pt}%
    \colorbox{brownshade}%
  }%
  \MakeFramed{\advance\hsize-\width\FrameRestore}%
  \noindent\hspace{-4.55pt}%
  \begin{adjustwidth}{}{7pt}%
    \vspace{2pt}\vspace{2pt}%
  }
  {%
    \vspace{2pt}
  \end{adjustwidth}\endMakeFramed%
}

% 问题块样式
\definecolor{greenshade}{rgb}{0.90,0.99,0.91}
\newenvironment{quesblock}{%
  \def\FrameCommand{%
    \hspace{0.01pt}%
    {\color{Green}\vrule width 2pt}%
    {\color{greenshade}\vrule width 00.1pt}%
    \colorbox{greenshade}%
  }%
  \MakeFramed{\advance\hsize-\width\FrameRestore}%
  \noindent\hspace{-4.55pt}%
  \begin{adjustwidth}{}{7pt}%
    \vspace{2pt}\vspace{2pt}%
  }
  {%
    \vspace{2pt}
  \end{adjustwidth}\endMakeFramed%
}

% 标记块
\newtcolorbox{markerblock}[1][]{enhanced,
  before skip=2mm,after skip=3mm,
  boxrule=0.4pt,left=5mm,right=2mm,top=1mm,bottom=1mm,
  colback=yellow!20,
  colframe=yellow!40!black,
  sharp corners,rounded corners=southeast,arc is angular,arc=3mm,
  underlay={%
    \path[fill=tcbcolback!80!black] ([yshift=3mm]interior.south east)--++(-0.4,-0.1)--++(0.1,-0.2);
    \path[draw=tcbcolframe,shorten <=-0.05mm,shorten >=-0.05mm] ([yshift=3mm]interior.south east)--++(-0.4,-0.1)--++(0.1,-0.2);
    \path[fill=yellow!80!black,draw=none] (interior.south west) rectangle node[white]{\Huge\bfseries !} ([xshift=4mm]interior.north west);
  },
drop fuzzy shadow,#1}

% 提示块
\definecolor{tipscolor}{rgb}{0.77,0.72,0.65}
\newtcolorbox{tipsblock}[2][]
{enhanced,breakable,
  left=12pt,right=12pt,
  coltitle=white,
  colbacktitle=tipscolor,
  attach boxed title to top left={yshifttext=-1mm},
  boxed title style={skin=enhancedfirst jigsaw,arc=1mm,bottom=0mm,boxrule=0mm},
  boxrule=1pt,
  colback=OldLace,
  colframe=tipscolor,
  sharp corners=northwest,
  title=\vspace{3mm}\textbf{#2},
  arc=1mm,
#1}

% 定理类彩色块
\newenvironment{thmblock}[1][\textbf{Theorem}]{
\begin{tcolorbox}[title=\textbf{#1}, colback=red!5,colframe=red!75!black]}{
\end{tcolorbox}}

\newenvironment{defblock}[1][\textbf{Definition}]{
\begin{tcolorbox}[colback = Emerald!10, colframe = cyan!40!black, title = \textbf{#1}]}{
\end{tcolorbox}}

\newenvironment{lemmablock}[1][\textbf{Lemma}]{
\begin{tcolorbox}[title=\textbf{#1},colback=SeaGreen!10!CornflowerBlue!10,colframe=RoyalPurple!55!Aquamarine!100!]}{
\end{tcolorbox}}

\newenvironment{propblock}[1][\textbf{Proposition}]{
  \begin{tcolorbox}
  [title = \textbf{#1}, colback=Salmon!20, colframe=Salmon!90!Black]}{
\end{tcolorbox}}

\newenvironment{colblock}[1][\textbf{Collary}]{
\begin{tcolorbox}[colback=JungleGreen!10!Cerulean!15,colframe=CornflowerBlue!60!Black,title = \textbf{#1}]}{
\end{tcolorbox}}

% Mathematical operator definitions
\DeclareMathOperator{\im}{im}
\DeclareMathOperator{\coker}{coker}
\DeclareMathOperator{\ind}{ind}

\usepackage[compat=1.1.0]{tikz-feynman}
\usepackage[all,cmtip]{xy}
\usepackage{tikz-cd}
\tikzfeynmanset{warn luatex=false}
\usetikzlibrary{positioning}
\usepackage{tikz}
\usetikzlibrary{external}
\tikzexternalize
\fancyhf{}
\fancypagestyle{plain}{
  \lhead{2025}
  \chead{\centering{Quantum Field Theory}}
  \rhead{\thepage\ of \pageref{LastPage}}
  \lfoot{}
  \cfoot{}
\rfoot{}}
\pagestyle{plain}
\newcommand{\Hom}{\operatorname{Hom}}

%-------------------basic info-------------------

\title{\textbf{Quantum Field Theory}}
\author{Lectured by Prof. Si Li and Noted by Xinyu Xiang}
\date{Jul. 2025}

%-------------------document---------------------

% ltex: enabled=true
\begin{document}
\maketitle

\textbf{Warning}: Lots of possible typos!!!!!!!!!!!!
\textbf{Notations}:
\begin{itemize}
  \item $ X$: a smooth manifold, usually a compact manifold.
  \item $ \mathcal{E}$: the space of fields, usually infinite dimensional.
  \item $ \mathrm{Conn}(P,X)$: the space of connections on a principal bundle $ P$ over $ X$.
  \item $ \text{Maps}(\Sigma, X)$: the space of maps from $\Sigma$ to $ X$.
  \item $ \Omega^{\bullet}(X)$: the space of differential forms on $ X$.
  \item $ \Omega^{\bullet}_{c}(X)$: the space of differential forms with compact support on $ X$.
  \item $ \mathrm{Vect}(M)$: the space of smooth vector fields on a manifold $ M$, which is Lie algebra of $ \mathrm{Diff}(M)$.
\end{itemize}

\section{Day I: Overall Discussion and Mathematical Preliminaries}

\subsection{Actions and Path Integrals}

Action $ S: \mathcal{E} \rightarrow \mathbf{k}$ where $ \mathcal{E}$ always has infinite dimension, and $ \mathbb{k}$ is a field (usually $ \mathbb{R}$ or $ \mathbb{C}$).

\begin{equation*}
  \text{QM in Imaginary Time} \xrightarrow{\text{ Brownian Motion}} \text{Wiener Measure on Phase Space}
\end{equation*}

\begin{equation*}
  \text{Asymptotic Analysis} \xrightarrow{\quad} \text{Perturbative Renormalisation Theory}
\end{equation*}

\begin{example} Some Examples of Classical Field Theories
  \begin{enumerate}[(a)]
    \item Scalar Field Theory $ \mathcal{E} = C^{\infty }(X)$
    \item Gauge Theory $ \mathcal{E} = \mathrm{Conn}(P,X)$
    \item $ \sigma$ Model $ \mathcal{E} = \text{Maps}(\Sigma, X)$
    \item Gravity $ \mathcal{E} = \text{Metrics}(X)$ (Better descriptions does not depend on the background)
  \end{enumerate}
\end{example}

\subsection{Observables}

Observables are functions on the space of fields, i.e. $ \mathcal{O} \in C^{\infty }(\mathcal{E})$.

\begin{example}[field theory]
  \begin{enumerate}[(a)]
    \item   Consider $ X = pt$, thus $ \mathcal{E} = \mathbb{R}^{n}$ for example.
    \item $ \dim X > 0$, the new algebraic structure arise form topological structures of $ X$.
  \end{enumerate}
\end{example}

The Key Point is: Capture the data of open sets of $ X$ $\longrightarrow$ Consider the observables supported on open set $ U$  of $ X$ denoted by $ \mathrm{Obs}(U)$ where $ U$ is an open set of $ X$.

Local data captures the open sets of $ X$. The relations between open sets captures the global data of $ X$ $\longrightarrow$ The algebraic structure of the observables is a sheaf of $ X$.
\begin{equation*}
  \bigsqcup_{i} U_{i} \longrightarrow \bigotimes_{i} \mathrm{Obs}(U_i)
\end{equation*}
Which implies OPE in physics and factorization algebra in mathematics.

Higher product in QFT: The generalization of products of algebra ('products in any direction instead of left and right') e.g. QM gives only left and right module of an algebra; OPE has products in various directions.

Consider the $\dim X = 2$ case in detailed
\begin{example}[Holomorphic/Chiral Field Theory]
  Various angle of product $ A(w) B(z)$  could be denoted by the time of $ A(w)$ rotations around $ B(z)$, which could be captured by the Fourier mode of $ A(w)$, thus one can have
  \begin{equation*}
    A(w) B(z) = \sum_{m \in \mathbb{Z}} \frac{( A_{(m) B(z)})}{(z - w)^{m+1}}
  \end{equation*}
  which is the Chiral algebra due to Beilinson and Drinfeld and associated with the Dolbeault cohomology $ H^{0}_{\bar{ \partial }}(\Sigma^{2} - \Delta) \cong \mathbb{C}((z^{m}))$, where $ \Sigma^{2}$ is the complex surface and $ \Delta$ is the diagonal of $ \Sigma^{2}$. The higher structure could be captured by the higher cohomology $ H^{p}_{\bar{ \partial }}(\Sigma^{2} - \Delta)$, which is the higher chiral algebra associated to the derived holomorphic section.
\end{example}

\subsection{De Rham Cohomology}

Chain of differential forms $ \Omega^{\bullet}(X)$
\begin{equation}\label{eq:deRhamChain}
  \Omega^{\bullet}(X) = \left( \cdots \xrightarrow{\mathrm{d}} \Omega^{n-1}(X) \xrightarrow{\mathrm{d}} \Omega^{n}(X) \xrightarrow{\mathrm{d}} \Omega^{n+1}(X) \xrightarrow{\mathrm{d}} \cdots \right)
\end{equation}
where $ \mathrm{d} $ is the exterior derivative, and $ \Omega^{n}(X)$ is the space of $ n$-forms on $ X$. The general construction of differential forms could be constructed over open set $ U$ by
\begin{equation*}
  \Omega^{n}(U) = \bigoplus_{1 \le i_1 \le \cdots \le i_n \le n} C^{\infty }(U) \mathrm{d} x^{i_1} \wedge \cdots \wedge \mathrm{d} x^{i_n}
\end{equation*}
where one can prove that $ \mathrm{d} ^{2} = 0$ and thus $\left( \Omega^{\bullet}(U), \mathrm{d} \right)$  is a cochain complex. The cohomology of it is called the de Rham cohomology $ H^{\bullet}(X)$.
\begin{proposition}
  The definition of de Rham cohomology does not depend on the choice of the open set $U$ and the choice of the coordinate system i.e. it is intrinsic $\longrightarrow$ we can define the de Rham cochain complex on smooth manifold $ X$.
\end{proposition}
\begin{proof}
  Consider
\end{proof}

\begin{definition}[de Rham Cohomology on Compact Support]
  Let $ X$ be a smooth manifold, then the de Rham cohomology on compact support is defined as
  \begin{equation}
    H^{\bullet}_{c}(X) = H^{\bullet}(\Omega^{\bullet}_{c}(X),\mathrm{d})
  \end{equation}
  where $ \Omega^{\bullet}_{c}(X)$ is the space of differential forms with compact support.
\end{definition}

\begin{theorem}[Stokes' Theorem]
  Let $ X$ be a smooth manifold with boundary, then for any $ \omega \in \Omega^{n}(X)$, we have
  \begin{equation*}
    \int_{X} \mathrm{d} \omega = \int _{\partial X} \omega
  \end{equation*}
  which connects the local data $ \mathrm{d} \Omega^{\bullet}(X)$ and the global data $\partial X$.
\end{theorem}
\begin{theorem}[Poincaré Lemma]
  \begin{equation*}
    H^{p}(\mathbb{R}^{n}) =
    \begin{cases}
      \mathbb{R} & p = 0 \\
      0 & p > 0
    \end{cases}, \quad
    H^{p}_{c}(\mathbb{R}^{n}) =
    \begin{cases}
      0 & p <0 \\
      \mathbb{R} & p = n \\
    \end{cases}
  \end{equation*}
  Generator: $ H^{p}(\mathbb{R}^{n})$ $\rightarrow $ constant function, $ H^{p}_{c}(\mathbb{R}^{n})$ $\rightarrow $ a compact support function $ \alpha = f(x) \mathrm{vol}_{n}$, and $ \int _{\mathbb{R}^{n}} \alpha = 1$.
\end{theorem}
\begin{proof}

\end{proof}
Important: An \emph{Integration} arises from the de Rham cohomology!
\begin{observation}
  \begin{enumerate}[(1)]
    \item If $ \alpha = \mathrm{d} \beta$ where $ \beta \in \Omega_{c}^{n-1}(X)$, then $ \int _{X} \alpha = 0$, thus the generator is $ \alpha$ whose integral is non-zero.
    \item \textbf{Dual Site}: Integration could be captured by the cohomology
      \begin{equation*}
        \int _{\mathbb{R}^{n}} \leftrightarrow H^{n}_{c}(\mathbb{R}^{n}) \cong \mathbb{R}
      \end{equation*}
      Path integral could be interpreted as the integration over $ \mathcal{E}$, which leads to consider the cohomology of it.
  \end{enumerate}
\end{observation}

\subsection{Cartan Formula}

Vector fields could act on smooth functions via
\begin{equation*}
  V(f) = V^{i} \frac{\partial f}{\partial x^{i}} = \frac{\mathrm{d} }{\mathrm{d} t} f(\varphi_{t}(x)) \bigg|_{t=0} = \frac{\mathrm{d} }{\mathrm{d} t} \varphi_{t}^{*} f(x) \bigg|_{t=0}
\end{equation*}
Such an action could be extended to differential forms by
\begin{equation*}
  \mathrm{Vect}(M) \ni V : \alpha \mapsto \mathcal{L}_{V} \alpha = \frac{\mathrm{d} }{\mathrm{d} t} \varphi_{t}^{*} \alpha \bigg|_{t=0}
\end{equation*}
which has the property $ \mathcal{L}_{V}(\alpha \wedge \beta) = \mathcal{L}_{V} \alpha \wedge \beta + \alpha \wedge \mathcal{L}_{V} \beta$, which implies that the Lie derivative is a derivation on the algebra of differential forms with degree $ 0$. And we have contraction $ \iota_{V}$ which is a derivation of degree $ -1$ on the algebra of differential forms.
\begin{equation*}
  \mathcal{L}_{V}  = \mathrm{d} \iota_{V} + \iota_{V} \mathrm{d}
\end{equation*}
Lie derivative is homotopy trivial i.e. chain homotopic.

\subsubsection{Proof of Poincaré Lemma}
Use Cartan Formula, one can proof Poincaré Lemma.

\begin{proof}
  Rescaling invariance of $ \mathbb{R}^{n}$ leads to the Euler vector field $ E = x^{i} \frac{\partial }{\partial x^{i}}$. One can consider the associated diffeomorphism $ \varphi_{t}$, where we assume $ \varphi_{0} = 1$ and thus $ \varphi^{*}_{-\infty } \alpha = 0$, thus the closed form $ \alpha$ could be rewritten as
  \begin{equation*}
    \begin{aligned}
      \alpha = & \varphi^*_{0} \alpha - \varphi^*_{-\infty } \alpha \\
      = & \int_{-\infty }^{0} \frac{\mathrm{d} }{\mathrm{d} t} \varphi_{t}^{*} \alpha \mathrm{d} t \\
      = & \int_{-\infty }^{0} \mathcal{L}_{E} (\varphi^{*}_{t} \alpha )\mathrm{d} t \\
    \end{aligned}
  \end{equation*}
  using the Cartan formula and $ \mathrm{d} \varphi^{*} = \varphi^{*} \mathrm{d} $, we have
  \begin{equation*}
    \alpha = \mathrm{d} \int_{-\infty }^{0} \varphi^{*}_{t} \iota_{E} \alpha \, \mathrm{d} t = \mathrm{d} \beta,
  \end{equation*}
  thus, the closed form $ \alpha$ is exact, which implies that the de Rham cohomology $ H^{p}(\mathbb{R}^{n})$ is trivial for $ p>0$. The same idea could be applied to the de Rham cohomology on compact support $ H^{p}_{c}(\mathbb{R}^{n})$.
\end{proof}

\section{Day II: Classical Field Theory}

Assume $ \mathcal{E} = \Gamma(E,X)$ i.e. a section of a bundle $ E \rightarrow X$, where $ X$ is oriented manifold.
And the action would be written as $ S[\phi] = \int _{X} \mathcal{L}[\phi(x)]$ where $ \phi \in \mathcal{E}$.
Lagrangian $ \mathcal{L}$ satisfies:
\begin{enumerate}[(a)]
  \item built up by jets of $ \phi$ (locality);
  \item valued in $ n$ form on $ X$ (oriented).
\end{enumerate}
The solution of Euler-Lagrange equation forms $ \mathrm{Crit}(S)$, which denote the critical locus of the action $S$.

\subsection{Examples}

\begin{example}[Phase Space Quantum Mechanics]
  Consider $ X = \mathbb{R}$, then $ \mathcal{E} = \mathbb{R}^{2n}$, and the action is
  \begin{equation*}
    \begin{aligned}
      S[\phi] = & \int _{\mathbb{R}^{2n}} p \mathrm{d} q - H(q,p) \mathrm{d} t
      = & \int \left[ p \dot{q} - H \right]\mathrm{d} t
    \end{aligned}
  \end{equation*}
  where $ H$ is the Hamiltonian. The Euler-Lagrange equation would become $ \mathrm{d} H = - \iota_{x_{*} \partial} \omega$, where $ x: \mathbb{R}\rightarrow \mathcal{E}$.
\end{example}

\begin{example}[Scalar Field Theory]
  Consider $(X,g)$ a Riemann Manifold, then $ \mathcal{E} = C^{\infty }(X)$, and the action is
  \begin{equation*}
    S[\phi] = \int _{X} \left[ \frac{1}{2} \left| \nabla \phi \right|^{2} + V(\phi) \right] \mathrm{d} \mathrm{vol}_{g}
  \end{equation*}
  where $ V(\phi)$ is a potential function, and $ \mathrm{d} \mathrm{vol}_{g} = \sqrt{\left| g \right|} \mathrm{d}^{d} x$. Assume $\partial X = \empty$, then the Euler-Lagrange equation is
  \begin{equation*}
    \Delta \phi = \frac{\partial V}{\partial \phi}
  \end{equation*}
  where $ \Delta f = \frac{1}{\sqrt{\left| g \right|}} \partial_{i}\left( \sqrt{\left| g \right|} g^{ij} \partial_{j} f\right)$.
\end{example}

\begin{example}[Chern-Simons Theory]
  Consider $ X$ a 3-manifold and $ \mathfrak{g}$ a semisimple Lie algebra. Denote $ P$ is a principal $ \mathfrak{g}$-bundle over $ X$, then the space of fields is $ \mathcal{E} = \mathrm{Conn}(P,X)$. Assume $ \mathfrak{g}$ is equipped with a non-degenerate invariant bilinear form $ \langle \cdot, \cdot \rangle$ (Killing form), then the action is
  \begin{equation*}
    \mathrm{CS}[A] = \int _{X} \frac{1}{2} \langle A, F_{A} \rangle + \frac{1}{6} \langle A, [A,A] \rangle,
  \end{equation*}
  and the Euler-Lagrange equation encoded by the flat connection $ F_{A} = 0$.
\end{example}

\subsection{Symmetry (1)}

\subsubsection{Global Symmetry and Noether's Theorem}

Consider a classical action $ S: \mathcal{E} \rightarrow \mathbb{R}$ with a group action $ G \curvearrowright \mathcal{E}$ s.t. $ S[g(\phi)] = S[\phi]$. Then $ G$ would become a global symmetry of the action $ S$.

Consider the continuous symmetry i.e. $ G$ is a Lie group, then the infinitesimal action of $ G$ on $ \mathcal{E}$ is given by a vector field $ V \in \mathrm{Vect}(\mathcal{E})$, which satisfies
\begin{equation*}
  \delta_{V^{\alpha}} \phi = V^{\alpha} (\phi),
\end{equation*}
thus the variation of the Lagrangian is
\begin{equation*}
  \delta_{V^{\alpha}} \mathcal{L} = \mathrm{d} K_{\alpha},
\end{equation*}
where $ K_{\alpha}$ is an $ n-1$ form. Furthermore, one can use the Euler-Lagrange equation, and it's boundary contribution to obtain
\begin{equation*}
  \delta_{V^{\alpha}} \mathcal{L} \xrightarrow{\mathrm{EL} = 0} \mathrm{d} \iota_{V^{\alpha}} \Theta = \mathrm{d} K_{\alpha},
\end{equation*}
thus one have the Noether's current
\begin{equation}\label{eq:NoetherCurrent}
  J_{\alpha} = \iota_{V^{\alpha}}\Theta - K_{\alpha}, \quad \mathrm{d} J_{\alpha} + EL[\phi] V_{\alpha} = 0,
\end{equation}
which is an $ n-1$ form on $ X$ and satisfies $ \mathrm{d} J_{\alpha}\big|_{\mathrm{Crit}(S)} = 0$ while the Euler-Lagrangian equation is satisfied.
If we consider $ Y_1, Y_2 \subset X$ is codimension $ 1$ (hyper)surface, which are homologous by $ \Sigma$, then we have
\begin{equation*}
  \int _{Y_1} J_{\alpha} - \int _{Y_2} J_{\alpha} = \int _{\Sigma} \mathrm{d} J_{\alpha} = 0, \quad \phi \in \mathrm{Crit}(S),
\end{equation*}
and the integration over $ J_{\alpha}$ is independent of the choice of the hyper surface, thus we can define the Noether charge as the integration over $ J_{\alpha}$ on a hyper surface $ Y$\footnote{In physics, one always consider the Noether current which is the Hodge dual of $J_{\alpha}$.}.

There is an alernative way to define the Noether current, which is more suitable for practical use. In brief, on can consider the 'gauged' symmetry which would promote $ \epsilon$ to become a field $ \epsilon(x)$, and the variation of the action could be computed by integrating by parts, finally one can obtain
\begin{equation*}
  \delta_{V^{\alpha}} S = \int _{X} - \epsilon(x) \mathrm{d} \hat{J}_{\alpha},
\end{equation*}
and $ \hat{J}$ would become the Noether current which satisfies \eqref{eq:NoetherCurrent} so that $ \hat{J}_{\alpha}$ is identical to $ J_{\alpha}$ up to an exact form.

\section{Day III: Breaking}

\section{Day IV: Symmetry (2)}

First, we will consider finite dimensional case. We consider $ G$ as a finite dimensional Lie group, $ \mathfrak{g} $ is the Lie algebra of $ G$ and $ W$ is finite dimensional representation of $ G$.

\subsection{Chevalley-Eilenberg Cohomology}

Consider $ \mathfrak{g}^{*} \equiv \Hom(\mathfrak{g}, \mathbb{K})$. Consider the exterior algebra
\begin{equation*}
  \bigwedge \mathfrak{g}^{*} = \bigoplus_{p=0}^{\infty } \bigwedge^{p} \mathfrak{g}^{*}.
\end{equation*}
Assume the basis of $ \mathfrak{g}$ is $\left\{ e_1,\cdots ,e_{n} \right\}$ and of $ \mathfrak{g}^{*}$ is $\left\{ c^{1},\cdots ,c^{n} \right\}$, which satisfies $ c_{\alpha} c_{\beta} = - c_{\beta} c_{\alpha}$. Thus, one could identify the algebra above as a free object in the category of differential graded algebra, which is a ring equipped with anti-commute generators
\begin{equation*}
  \bigwedge \mathfrak{g}^{*} = \mathbb{K}[c^{1},\cdots ,c^{n}].
\end{equation*}

Consider the Lie algebra over $ \mathfrak{g}$, which equipped with commutator $[\cdot, \cdot] : \wedge^{2} \mathfrak{g} \rightarrow \mathfrak{g}$. One the dual side, one would introduce a differential operator $ \mathrm{d} : \mathfrak{g}^{*} \rightarrow \mathfrak{g}^{*}$, and we can extend it to the exterior algebra $ \bigwedge \mathfrak{g}^{*}$ by
\begin{enumerate}[(1)]
  \item Under the level of generators, we have $ \mathrm{d} _{\mathrm{CE}} : \mathfrak{g}^{*} \rightarrow \wedge^{2} \mathfrak{g}^{*}$;
  \item Using the Leibniz rule, we can extend it to the exterior algebra $ \bigwedge \mathfrak{g}^{*}$ by
    \begin{equation*}
      \mathrm{d} _{\mathrm{CE}} : a \wedge b \mapsto \mathrm{d} _{\mathrm{CE}} a \wedge b + (-1)^{\deg a} a \wedge \mathrm{d} _{\mathrm{CE}} b,
    \end{equation*}
\end{enumerate}
and thus we have a differential graded algebra $ \left( \bigwedge \mathfrak{g}^{*}, \mathrm{d} _{\mathrm{CE}} \right)$, which is called the Chevalley-Eilenberg complex.

Under the choice of basis above, we have $[e_{\alpha} , e_{\beta}] = f^{\gamma}_{\alpha \beta} e_{\gamma}$, which would lead to the derivation on the dual side
\begin{equation*}
  \mathrm{d} _{\mathrm{CE}} c^{\alpha} = \frac{1}{2} f^{\alpha}_{\beta \gamma} c^{\beta} \wedge c^{\gamma} \equiv \frac{1}{2}f^{\alpha}_{\beta \gamma} c^{\beta}c^{\gamma}.
\end{equation*}
Using the Leibniz rule, we can extend it to the exterior algebra $ \bigwedge \mathfrak{g}^{*}$. Using the Jacobi identity, one can prove that $ \mathrm{d} _{\mathrm{CE}}^{2} = 0$ (left as exercise), thus we have a cochain complex $\left( \bigwedge \mathfrak{g}^{*}, \mathrm{d} _{\mathrm{CE}} \right)$ which is a differential graded algebra (dga),
where the generator $ c^{\alpha} $ is called the 'ghost field' in physics, the degree is 'ghost number' and $ \mathrm{d} _{\mathrm{CE}}$ is BRST operator.
\begin{proof}
  Consider $ \mathrm{d} _{\mathrm{CE}}^{2}$ acts on $ c^{\alpha}$, the higher structure could be derived from Leibniz's rule.
  \begin{equation*}
    \begin{aligned}
      \mathrm{d} ^{2}_{\mathrm{CE}} c^{\alpha} = & \frac{1}{2} f^{\alpha}_{\beta \gamma} \left[ \frac{1}{2} f^{\beta}_{\rho \lambda} c^{\rho} c^{\lambda} c^{\gamma} - \frac{1}{2} f^{\gamma}_{\rho \lambda} c^{\beta} c^{\rho} c^{\lambda} \right] \\
      = & - \frac{1}{2} f^{\alpha}_{\gamma \beta} f^{\beta}_{\rho \lambda} c^{\rho} c^{\lambda} c^{\lambda} \\
      = & \frac{1}{12} f_{\beta [\gamma}^{\alpha} f_{\rho \lambda]}^{\beta} c^{\rho} c^{\lambda} c^{\lambda} \\
      = & 0
    \end{aligned}
  \end{equation*}
\end{proof}

Let $ M$ be a $ \mathfrak{g}$ representation where $ \rho : \mathfrak{g} \rightarrow \operatorname{End}(W)$ satisfies
\begin{equation*}
  \rho(a) \rho(b) m - \rho(b) \rho(a) m = \rho([a,b]) m, \quad a,b \in \mathfrak{g}, m \in M.
\end{equation*}
Consider the free $\bigwedge^{\bullet} \mathfrak{g}^{*}$-module generated by $ M$:
\begin{equation*}
  \bigwedge\nolimits^{\bullet} \mathfrak{g}^{*} \otimes M,
\end{equation*}
there is a natural extension of the Chevalley-Eilenberg differential $ \mathrm{d} _{\mathrm{CE}}$ on it, which is defined by
\begin{enumerate}[(1)]
  \item $ \mathrm{d} _{\mathrm{CE}} : M \rightarrow g^{*}\otimes M$ is dual of $ g \otimes M \xrightarrow{\rho} M$;
  \item $ \mathrm{d} _{\mathrm{CE}}(a\otimes m) : \mathrm{d} _{\mathrm{CE}}(a) \otimes m + (-1)^{|a|} a \wedge \mathrm{d} _{\mathrm{CE}} m$
\end{enumerate}
where we can prove that $ \mathrm{d} _{\mathrm{CE}}^{2} = 0$, and thus we have a cochain complex $\bigwedge^{\bullet} \mathfrak{g}^{*} \otimes M$.

We denote $\wedge^{p} \mathfrak{g}^{*} \otimes M$ be $ C^{p}(\mathfrak{g}^{*}, M)$, then we would find that it is $ C^{p}(\mathfrak{g}^{*})$-module, i.e.
\begin{equation*}
  C^{p}(\mathfrak{g}^{*}) \otimes C^{q}(\mathfrak{g}^{*}, M) \ni a \otimes v \mapsto a \wedge v \in C^{p+q}(\mathfrak{g}^{*}, M),
\end{equation*}
which is compatible with derivation
\begin{equation*}
  \mathrm{d} _{\mathrm{CE}} (a\wedge v) = \mathrm{d} _{\mathrm{CE}} a \wedge v + (-1)^{\left| a \right|} a \wedge \mathrm{d} _{\mathrm{CE}} v,
\end{equation*}
where $ m \in M$ and $ a \in \wedge^{\bullet} \mathfrak{g}^{*}$. The derivation could be written explicitly with basis $ a_{k}$ of $ M$, and it's dual basis $ b^{k}$:
\begin{equation*}
  \mathrm{d} _{\mathrm{CE}} = \left( \rho_{\alpha} \right)^{k}_{i} b^{i} a_{k} c^{\alpha} + \frac{1}{2} f^{\alpha}_{\beta \gamma} c^{\beta} c^{\gamma} a_{\alpha},
\end{equation*}
which could be easily verified that $ \mathrm{d} ^{2}_{\mathrm{CE}} = 0$.
\begin{proof}
  There is a general way to prove $ \mathrm{d} _{\mathrm{CE}}^{2} = 0$, which is to note that, under the dual transformation, one have identity $\left< \mathrm{d} _{\mathrm{CE}} \varphi_1 , c_1 m_1 \right> = \left< \varphi_1, \rho(c_1) m_1 \right>$, $\left< \mathrm{d} _{\mathrm{CE}} \varphi_2, c_{1}\wedge c_{2} \right> = \left< \varphi_2, [c_1, c_2] \right>$ and Leibniz's law, so that:
  \begin{equation*}
    \left< \mathrm{d} ^{2}_{\mathrm{CE}} \varphi, c_1 \wedge c_2 \otimes m_1 \right> = \left< \varphi , \rho([c_1, c_2]) m + (-1)( \rho(c_1) \rho(c_2) m + \rho(c_2) \rho(c_1) m ) \right> \xrightarrow{\text{$ \mathfrak{g}$ rep.}} 0,
  \end{equation*}
  \begin{equation*}
    \left< \mathrm{d} ^{2}_{\mathrm{CE}} \varphi, c_1 \wedge c_2 \wedge c_3 \right> \xrightarrow{\text{Jacobian}} 0,
  \end{equation*}
  where we note that the dual of $ c_1 \wedge c_2$ has degree $ 1$ graded.
\end{proof}

\subsection{Differential Graded Lie Algebra}

We define a $ \mathbb{Z}$-graded vector space
\begin{equation*}
  W = \bigoplus_{n \in \mathbb{Z}} W_{n},
\end{equation*}
where $ W_{n}$ is degree of $ n$ component.
\begin{enumerate}
  \item \textbf{Degree Shift}: $ W[n]_{m} \equiv W_{n+m}$;
  \item \textbf{Dual}: $ W^{*}$ denote the linear dual of $ W$
    \begin{equation*}
      W^{*}_{n} = \Hom(W_{-n}, \mathbb{K});
    \end{equation*}
  \item \textbf{Symmetry and Anti-Symmetry}: $ \mathrm{Sym}^{\otimes n}(V) = V^{\otimes n} / \sim $ where $a \otimes b \sim (-)^{|a||b|} b \otimes a$,
    and $ \bigwedge\nolimits^{V} = V^{\otimes n} / \sim $ where $ a \otimes b \sim (-1)^{|a||b|+1} b \otimes a$;
\end{enumerate}
which has a natural isomorphism between $\bigwedge\nolimits^{m} \left( V[1] \right) $ and $\mathrm{Sym}^{m} (V)[m]$
\begin{proposition}
  Let $ V$ be a dga, then:
  \begin{equation*}
    \bigwedge\nolimits^{m} \left( V[1] \right) \cong \mathrm{Sym}^{m} (V)[m].
  \end{equation*}
\end{proposition}
\begin{proof}
  Consider the subspace generated by ideals
  \begin{equation*}
    a \otimes b \sim (-1)^{(|a|+1)(|b| + 1) + 1} b \otimes a = (-1)^{|a||b| + |a| + |b|} b \otimes a, \quad a,b \in V[1],
  \end{equation*}
  \begin{equation*}
    a \otimes b \sim (-1)^{|a||b|} b \otimes a, \quad a,b \in V,
  \end{equation*}
  where $\left| a \right|$ is the degree of $ a$ in $ V$, thus the total degree in $ V[1]$ is $\left| a \right| + \left| b \right| + 2$.
  The element in the left-hand side is
  \begin{equation*}
    \frac{1}{n!} \left( a_1a_2 \cdots a_n + (-1)^{(\left| a_1 \right|+1) \left( \sum_{i=2}^{n} \left| a_i \right| + n-1\right) + n - 1} a_2 \cdots a_n a_1  + \cdots \right) \in \bigwedge\nolimits^{n} \left( V[1] \right),
  \end{equation*}
  and the element in the right-hand side is
  \begin{equation*}
    \frac{1}{n!} \left( a_1a_2 \cdots a_n + (-1)^{\left| a_1 \right| \sum_{i=2}^{n} \left| a_i \right| } a_2 \cdots a_n a_1  + \cdots \right) \in \mathrm{Sym}^{n} (V)[n].
  \end{equation*}
  Consider the shuffle map
  \begin{equation*}
    a_1 \otimes a_2 \otimes \cdots \otimes a_n \rightarrow a_n \otimes a_{n-1} \otimes \cdots \otimes a_{1},
  \end{equation*}
  the overall sign in $ \mathrm{Sym}^{m}(V)[m]$ and $\bigwedge^{m}(V[1])$ is the same, which is
  \begin{equation*}
    (-1)^{\sum_{1 \le i < j \le n} \left| a_i \right| \left| a_j \right|} = (-1)^{\sum_{1 \le i < j \le n} \left| a_i \right|\left| a_j \right| + 2 \sum_{i=1}^{n} \left| a_i \right|},
  \end{equation*}
  where the first term is the sign of the anti-symmetry monomials and the second term is the sign of the symmetry monomials.

\end{proof}

\begin{definition}[Differential Graded Lie Algebra]
  A DGLA is a $\mathbb{Z}$-graded space
  \begin{equation*}
    g = \bigoplus_{m \in \mathbb{Z}} g_m
  \end{equation*}
  together with bilinear map $[\cdot,\cdot]: g \otimes g \to g$ satisfying
  \begin{enumerate}
    \item (graded bracket) $[g_\alpha, g_\beta] \subset g_{\alpha+\beta}$, $[\cdot,\cdot] \in \text{Hom}(g \otimes g, g)$,
    \item (graded skew-symmetry) $[a,b] = -(-1)^{|a||b|}[b,a]$ \quad $\left([\cdot,\cdot]: \wedge^2 g \to g\right)$,
    \item (graded Jacobi Identity) $[[a,b],c] = [a,[b,c]] - (-1)^{|a||b|}[b,[a,c]]$,
  \end{enumerate}
  with a degree 1 map $d: g \to g$ (i.e., $d: g_\alpha \to g_{\alpha+1}$) satisfying $ \mathrm{d} ^{2} = 0$ and
  \begin{enumerate}
      \setcounter{enumi}{3}
    \item (graded Leibniz rule) $d[a,b] = [da,b] + (-1)^{|a|}[a,db]$.
  \end{enumerate}
\end{definition}

% generated by claude 3.7
\begin{example}[de-Rham + Lie = DGLA]
  Let $X$ be a manifold, $\mathfrak{g}$ a Lie algebra.
  \begin{itemize}
    \item $(\Omega^{\bullet}(X), d)$ de Rham complex,
    \item $(\Omega^{\bullet}(X) \otimes \mathfrak{g}, d, [\cdot,\cdot])$ is DGLA,
    \item $\Omega^p(X) \otimes \mathfrak{g}$ : degree $p$ component,
    \item $d: \Omega^p \otimes \mathfrak{g} \to \Omega^{p+1} \otimes \mathfrak{g}$ de Rham, $d(\alpha \otimes h) = d\alpha \otimes h$,
    \item $[\cdot,\cdot]$ induced from $\mathfrak{g}$,
    \item Let $ \alpha_{1,2} \in \Omega^{\bullet}(X)$, $ h_{1,2} \in \mathfrak{g}$, then $[\alpha_1 \otimes h_1, \alpha_2 \otimes h_2] = \alpha_1 \wedge \alpha_2 \otimes [h_1, h_2]$,
  \end{itemize}
  $\leadsto$ DGLA in Chern-Simons theory.
\end{example}

\begin{example}[Dolbeault + Lie = DGLA]
  Let $X$ be a complex manifold. Let
  \begin{itemize}
    \item $(\Omega^{0,*}(X), \bar{\partial})$ Dolbeault Complex,
    \item $(\sum_{\bar{i}_1,...,\bar{i}_p} \varphi_{\bar{i}_1...\bar{i}_p} d\bar{z}^{\bar{i}_1} \wedge ... \wedge d\bar{z}^{\bar{i}_p})$ where $\bar{\partial} = d\bar{z}^{\bar{i}} \frac{\partial}{\partial \bar{z}^i}$,
    \item $T_X \otimes_{\mathbb{C}} \mathbb{C} = T_X^{1,0} \oplus T_X^{0,1}$, where we could choose the basis as
      \begin{equation*}
        \text{Span}\{\frac{\partial}{\partial z^i}\}, \quad \text{Span}\{\frac{\partial}{\partial \bar{z}^i}\},
      \end{equation*}
  \end{itemize}
  which leads that $(\Omega^{0,*}(X, T_X^{1,0}), \bar{\partial}, [\cdot,\cdot])$ is a DGLA.

  Explicitly, let $\{z^i\}$ be local holomorphic coordinates.
  $\alpha \in \Omega^{0,p}(X, T_X^{1,0})$ takes the form
  \begin{equation*}
    \alpha = \sum_{i,\bar{J}} \alpha_{\bar{J}}^i d\bar{z}^{\bar{J}} \otimes \frac{\partial}{\partial z^i}, \quad d\bar{z}^{\bar{J}} = d\bar{z}^{\bar{j}_1} \wedge ... \wedge d\bar{z}^{\bar{j}_p},
  \end{equation*}
  \begin{equation*}
    \bar{\partial}\alpha = \sum_i \bar{\partial}\alpha_{\bar{J}}^i d\bar{z}^{\bar{J}} \otimes \frac{\partial}{\partial z^i} = \sum_i \frac{\partial \alpha_{\bar{J}}^i}{\partial \bar{z}^k} d\bar{z}^k \wedge d\bar{z}^{\bar{J}} \otimes \frac{\partial}{\partial z^i}.
  \end{equation*}
  Let $\alpha = \sum_i \alpha_{\bar{J}}^i d\bar{z}^{\bar{J}} \otimes \frac{\partial}{\partial z^i}$ and $\beta = \sum_m \beta_{\bar{M}}^i d\bar{z}^{\bar{M}} \otimes \frac{\partial}{\partial z^i}$.
  The Lie bracket is
  \begin{equation*}
    [\alpha, \beta] = \sum_i \left(\alpha_{\bar{J}}^j \partial_j \beta_{\bar{M}}^i - \beta_{\bar{M}}^j \partial_j \alpha_{\bar{J}}^i\right) d\bar{z}^{\bar{J}} \wedge d\bar{z}^{\bar{M}} \otimes \frac{\partial}{\partial z^i}
  \end{equation*}
  On $\deg = 0$ component, this is the standard Lie bracket of $(1,0)$ vector fields. Finally, one can verify that $(\Omega^{0,*}(X, T_X^{1,0}), \bar{\partial}, [\cdot,\cdot])$ is a DGLA.
  \noindent $\leadsto$ Mathematics: Deformation of complex structures $\longleftrightarrow$ Physics: B-twisted topological string (Kodaira-Spencer gravity)
\end{example}

We can consider the Chevalley-Eilenberg complex for a DGLA $\left( \mathfrak{g}, \mathrm{d} , [~,~] \right)$.

\begin{definition}[Chevalley-Eilenberg Complex]
  For a DGLA $\left( \mathfrak{g}, \mathrm{d} , [~,~] \right)$, the Chevalley-Eilenberg complex is defined as
  \begin{equation*}
    C^{\bullet}(\mathfrak{g}) = \mathrm{Sym}^{\bullet}\left( g^{*}[-1] \right) = \bigwedge\nolimits^{\bullet} \mathbf{g}^{*} [-\bullet],
  \end{equation*}
  % TODO: understand the relation with Bg
  equipped with the CE differential $ \mathrm{d} _{\mathrm{CE}} = \mathrm{d}_1+\mathrm{d}_2$, where
  \begin{enumerate}[(1)]
    \item $ \mathrm{d}_1: \mathfrak{g}^{*}[-1] \rightarrow \mathfrak{g}^{*}[-1]$ is the dual of $ \mathrm{d} : \mathfrak{g} \rightarrow \mathfrak{g}$;
    \item $ \mathrm{d}_2 : \mathfrak{g}^{*}[-1] \rightarrow \mathrm{Sym}^{2} \left( \mathfrak{g}^{*}[-1] \right) \cong \bigwedge\nolimits^{2} \mathfrak{g}^{*}[-2]$ is the dual of $[~,~]: \bigwedge^{2} \rightarrow \mathfrak{g}$;
    \item (Graded Leibniz rule) The derivation extends to
      \begin{equation*}
        \mathrm{d} _{\mathrm{CE}} : \mathrm{Sym}(\mathfrak{g}^{*}[-1]) \rightarrow \mathrm{Sym}(\mathfrak{g}[-1])
      \end{equation*}
      via the graded Leibniz rule
      \begin{equation*}
        \mathrm{d} _{\mathrm{CE}}(ab) = \mathrm{d} _{\mathrm{CE}} a \, b + (-1)^{|a|} a \, \mathrm{d} _{\mathrm{CE}} b,
      \end{equation*}
      and satisfies $ \mathrm{d} _{\mathrm{CE}}^{2} = 0$.
  \end{enumerate}
\end{definition}
\begin{remark}
  If $ \mathfrak{g}$ degenerated to the ordinary Lie algebra, which would be 'bosonic' fields.
  However, the basic object to build CE complex for ordinary Lie algebra is 'fermionic' fields. So we need to impose $[-1]$ into the definition of CE complex of DGLA.
\end{remark}

\begin{definition}[DGLA-module]
  Let $\mathfrak{g}$ be a DGLA. A $\mathfrak{g}$-module is a cochain complex $(M, d_M)$ with bilinear map
  \begin{equation*}
    \mathfrak{g} \otimes M \rightarrow M
  \end{equation*}
  where $ C^{\bullet}(\mathfrak{g}, M) = \mathrm{Sym}\left( \mathfrak{g}^{*}[-1] \right) \otimes N$ satisfying
  \begin{enumerate}[(1)]
    \item $ \mathrm{d} _{M}$ is the dual of $\rho: \mathfrak{g}_n \otimes M_p \rightarrow M_{n+p}$,
    \item $ \mathrm{d} _{\mathfrak{g}}$ is the dual of $[~,~]: \rho(a) \rho(b) m - (-1)^{|a||b|} \rho(b) \rho(a) m = \rho([a, b]) \cdot m$,
    \item (Chevalley-Eilenberg differential) $ \mathrm{d} _{CE} = \mathrm{d} _{M} + \mathrm{d} _{\mathfrak{g}}$,
    \item (Leibniz's law) $\mathrm{d}_\mathrm{CE}(a \otimes m) = (\mathrm{d}_{\mathfrak{g}} a) m + (-1)^{|a|} a \mathrm{d}_M m$,
  \end{enumerate}
\end{definition}

\subsection{\texorpdfstring{Homotopic Lie Algebra ($L_{\infty}$ Algebra)}{ Homotopic Lie Algebra (L-infinity Algebra)}}

\subsubsection{Coderivation Side}

The original definition could be viewed as a homotopic generalization of the Lie algebra, which is a DGLA $ V$  with 'higher brackets' $ \mu_{n}: V^{\otimes n} \rightarrow V$, where the first term at the chain level formed a (co)chain complex i.e. $ \mu_1^{2} = 0$.
The higher brackets needed to satisfy some self-consistency conditions, which is so called 'homotopic Jacobian identity'.
At some low level $ n$, which could be written explicitly as
\begin{equation*}
  \mu_1 \mu_2(a,b)
  = -\,\mu_2(\mu_1 a, b)-(-1)^{|a|}\,\mu_2(a, \mu_1 b),
\end{equation*}
\begin{equation*}
  \begin{aligned}
    &\mu_1 \mu_3(a,b,c)
    + \mu_3(\mu_1 a, b, c)
    + (-1)^{|a|}\,\mu_3(a, \mu_1 b, c)
    + (-1)^{|a|+|b|}\,\mu_3(a, b, \mu_1 c)\\
    &\qquad
    = -\,\mu_2\bigl(\mu_2(a,b),c\bigr)
    - (-1)^{(|b|+|c|)|a|}\,\mu_2\bigl(\mu_2(b,c),a\bigr)
    - (-1)^{(|a|+|b|)|c|}\,\mu_2\bigl(\mu_2(c,a),b\bigr),
  \end{aligned}
\end{equation*}
where $ a, b, c \in V$ is the element of $ L_\infty$ algebra $ V$.

The infinite number of brackets could be rewritten into a more compact form via coalgebra, and it's coderivation. For this need, we introduce the graded algebra
\begin{equation*}
  S^{c} V = \bigoplus_{n=0}^{\infty} V^{\wedge n}[-n],
\end{equation*}
where we note that the monomial $ a_1 a_2\cdots a_n \in V^{\wedge n}$ satisfied
\begin{equation*}
  a_1 a_2 \cdots a_n = (-1)^{\left| a_i \right| \left| a_{i+1} \right|} a_1 a_2 \cdots a_{i+1} a_i \cdots a_n.
\end{equation*}
We introduce the coproduct $ \Delta : S^{c} V \rightarrow S^{c} V \otimes S^{c} V$, which is defined by
\begin{equation*}
  \Delta : a_1 \cdots a_n \mapsto \sum_{i=1}^{n} \sum_{\sigma \in \mathrm{Sh}(i,n)} (-1)^{\sigma} a_{\sigma(1)} \cdots a_{\sigma(i)} \otimes a_{\sigma(i+1)} \cdots a_{\sigma(n)},
\end{equation*}
where the shuffle map $ \mathrm{Sh}(i,n)$ is the set of all possible ways of permutations which satisfies $ \sigma(1) < \cdots \sigma(i)$ and $ \sigma(i+1) < \cdots < \sigma(n)$,
and the sign $ (-1)^{\sigma}$ is the sign of the permutation $\sigma$.
The coproduct is coassociative, i.e.
\begin{equation*}
  (\Delta \otimes Id) \Delta = (Id \otimes \Delta) \Delta.
\end{equation*}

\subsubsection{Derivation Side}

\section{Day V: Perturbation Theory}

\section{Day VI: UV Divergence}

\subsection{Perturbative Quantum Field Theory}

We would consider the perturbative theory of a scalar field theory, where $ \mathcal{E} = C^{\infty }(X)$, $ X = \mathbb{R}^{d}$ and the action is given by
\begin{equation*}
  S[\phi] = \int _{X} \left( \frac{1}{2} \phi \Box \phi + \frac{1}{2} m^{2} \phi^{2} \right) \mathrm{d}^{d} x,
\end{equation*}
where the observables could be defined as correlators, which could be defined as the expectation value of the product of fields
\begin{equation*}
  \left< \mathcal{O} \right> = \frac{\int \mathcal{D}[\phi] \, \mathcal{O} \exp \left[ - \frac{1}{\hbar}\int_{X} \mathrm{d}^{d} x \, \left( \frac{1}{2} \phi \Box \phi + \frac{1}{2} m^{2} \phi^{2} \right) \right] }{\int \mathcal{D}[\phi] \, \exp \left[ - \frac{1}{\hbar} \int_{X} \mathrm{d}^{d} x \, \left( \frac{1}{2} \phi \Box \phi + \frac{1}{2} m^{2} \phi^{2}  \right)\right] },
\end{equation*}
which could be computed by Wick's contraction and Green's function
\begin{equation*}
  \left( \Box + m^{2} \right) G(x,y) = \hbar \delta(x-y),
\end{equation*}
thus the observable $\left< \phi(x_1) \phi(x_2) \cdots \phi(x_n) \right>$ could be computed by
\begin{equation*}
  \left< \phi(x_1) \phi(x_2) \cdots \phi(x_2n) \right> = \hbar^{n} \sum_{\sigma \in \mathbb{S}_{2n}} G(x_{\sigma(1)}, x_{\sigma(2)}) G(x_{\sigma(3)}, x_{\sigma(4)}) \cdots G(x_{\sigma(2n-1)}, x_{\sigma(2n)}),
\end{equation*}
which has asymptotic expansion in the limit $ x-y \rightarrow \infty $:
\begin{equation*}
  G(x,y) \sim \frac{1}{\left| x - y \right|^{d-2}},
\end{equation*}
for $ d>2$. Such an asymptotic expansion would lead to the divergence of the observable, which is called ultraviolet (UV) divergence.

Consider the interaction term
\begin{equation*}
  I_3(\phi) = \int_{X} \mathrm{d}^{d} x \, \frac{\lambda_3}{3!} \phi^{3}, \quad
  I_4(\phi) = \int_{X} \mathrm{d}^{d} x \, \frac{\lambda_4}{4!} \phi^{4},
\end{equation*}
which would twist the observables to a new form which could be also computed by Feynman diagrams.

\subsection{Canonical Quantization}

In classical mechanics, one would consider the phase space $(M, \omega)$, where $ \omega$ is the symplectic form, which defined a symplectic structure on the phase space $ M$
\begin{equation*}
  \omega(V_{f}, V_{g}) = \left\{ f, g \right\},
\end{equation*}
where $ \iota_{V_f} \omega = \mathrm{d} f$. The deformation quantization would lead to a non-commutative, where the commutative product in $ C^{\infty }(M)$ is replaced by the Moyal product:
\begin{equation*}
  \left( \mathcal{A} = \mathbb{R}[x,y], \cdot, \left\{ ~, ~ \right\} \right) \xrightarrow{\text{Deformation Quantization}} \left( \mathcal{A}[[\hbar]], \star, [~,~] \right),
\end{equation*}
where the Moyal product is defined by
\begin{equation*}
  f \star g = f(x,p) \exp\left( \frac{\mathrm{i}\hbar}{2} \left( \overleftarrow{\partial_x} \overrightarrow{\partial_p} - \overleftarrow{\partial_p} \overrightarrow{\partial_x} \right) \right) g(x,p),
\end{equation*}
which is a non-commutative and associative product on the algebra $ \mathcal{A}[[\hbar]]$.
\begin{proof}[The Associativity of Moyal Product]
  TBD.
\end{proof}

We can derive the Moyal product from the path integral, where the action is
\begin{equation*}
  S[x,p] = - \mathrm{i} \int _{\mathbb{R}} \mathbb{P} \mathrm{d} \mathbb{X},
\end{equation*}
which is called \emph{topological quantum mechanics}, where the path integral is defined as:
\begin{equation*}
  \left< \mathcal{O} \right> = \frac{\int \mathcal{D}[\mathbb{X}, \mathbb{P}] \, \mathcal{O} \exp\left( -\frac{1}{\hbar} S[\mathbb{X} , \mathbb{P}] \right)}{\int \mathcal{D}[\mathbb{X}, \mathbb{P}] \, \exp\left( -\frac{1}{\hbar} S[\mathbb{X}, \mathbb{P}] \right)},
\end{equation*}
where the green's function could be defined as:
\begin{equation*}
  G(t_1,t_2) = \frac{1}{2} \sgn(t_1 - t_2) =
  \begin{cases}
    \frac{1}{2}, & t_1 > t_2, \\
    0, & t_1 = t_2, \\
    -\frac{1}{2}, & t_1 < t_2,
  \end{cases}
\end{equation*}
thus $\left< \mathbb{X}(t_1) \mathbb{P}(t_2)\right> = - \mathrm{i}\hbar G(t_1, t_2)$ and $\left< \mathbb{P}(t_1) \mathbb{X}(t_2)\right> = \mathrm{i} \hbar G(t_1 , t_2)$.

We can use the Moyal product to construct the algebra of observables, whose product could be interpreted as the OPE and derived from the path integral. Consider $ f, g \in \mathbb{R}[x,p]$, thus, near the classical solution $(x,p)$ for each function, one can express the path integral as
\begin{equation*}
  \left< f(x + \mathbb{X}(t_1), p + \mathbb{P}(t_1)) g(x+\mathbb{X}(t_2), p + \mathbb{P}(t_2)) \right> = f \star g.
\end{equation*}
\begin{proof}
  The process of Wick contraction could be realized by the operator
  \begin{equation*}
    \left( \frac{\partial }{\partial P_0^{L}}\right)_{ij} = \sum_{klpq} \mathrm{i} \hbar G_{kl}(t_i, t_{j}) \eta_{pq} \left( \frac{\partial }{\partial \left( \partial^{k} \phi_{p} \otimes \partial^{l} \phi_{q} \right)} \right)_{ij} \xrightarrow{f, g \in \mathbb{R}[x,p]} \mathrm{i} \hbar \sum_{pq} G(t_i, t_j) \omega_{pq} \left( \frac{\partial }{\partial \left( \phi_{p} \otimes \phi_{q} \right)} \right)_{ij},
  \end{equation*}
  where $ \phi = (\mathbb{X}, \mathbb{P})$ and $ \eta \equiv \omega$ gives the contribution of the correct sign in the Moyal product. The $ n$ links contraction would lead to the operator
  \begin{equation*}
    \text{$ n$ Links Wick Contraction} \leadsto \frac{1}{n!} \left( \frac{\partial }{\partial P^{L}_{0}} \right)^{n}_{ij},
  \end{equation*}
  thus the Feynman diagram could be rewritten as
  \begin{equation*}
    \left< f(x+\mathbb{X}(t_1), p + \mathbb{P}(t_1)) g(x + \mathbb{X}(t_2) , p + \mathbb{P}(t_2)) \right> = \mathrm{Mult} \left( \mathrm{e}^{\frac{\partial }{\partial P_{0}^{L}}} \left( f(\phi(t_1)) \otimes g(\phi(t_2))\right)\right),
  \end{equation*}
  where the $ \mathrm{Mult}$ is the multiplication of the algebra $ \mathcal{A}[[\hbar]]$. Write the definition of the contraction operator, we obtain OPE
  \begin{equation*}
    f(x,p) \star g(x,p) \sim f(x,p) \exp\left( \frac{\mathrm{i}\hbar}{2}\left( \overleftarrow{\partial_x} \overrightarrow{\partial_p} - \overleftarrow{\partial_p} \overrightarrow{\partial_x} \right) \right) g(x,p) + \cdots,
  \end{equation*}
  where the $ \cdots$ is the correction term in $ \hbar$ while $ f,g$ is not polynomial. If $ f, g \in \mathbb{R}[x,y]$ is polynomial, then the correction term is zero, and we have the Moyal product.
\end{proof}

% TODO: Hamiltonian as the deformation of the theory

\subsection{\texorpdfstring{Counter Term in Perturbative $\phi^{4}$}{ Counter Term in Perturbative phi^4}}

If one consider the tree level Feynman diagram, we could proof that there is no UV divergence. However, if one consider the loop level Feynman diagram, some problems would occur. For example, consider the 1-loop Feynman diagram of the scalar field theory, which is given by
\begin{equation*}
  \hbar^{2} \lambda^{2} \int _{(\mathbb{R}^{4})^{2}} \mathrm{d} x_1 \mathrm{d} x_2 \, \phi^{2}(x_1) \phi^{2}(x_2) \frac{1}{\left| x_1 - x_2 \right|^{4}},
\end{equation*}
where we assume $ d=4$, such an integral would be divergent $ \sim \log r$. The physical interpreter is considering the cutting off and set the coupling constants depending on the cutting of scaling.

After introducing the cutting off, the propagator would become
\begin{equation*}
  G(x,y) = \int \mathrm{d} ^{4} k \, \frac{\mathrm{e}^{\mathrm{i}k(x-y)}}{k^{2}} \xrightarrow{\text{Cutting Off}} \int _{\Lambda_{0} \le k \le \Lambda_1} \mathrm{d} ^{4} k \, \frac{\mathrm{e}^{\mathrm{i}k(x-y)}}{k^{2}},
\end{equation*}
another way to interpret the cutting off is to consider the heat kernel cutting off, which is given by
\begin{equation*}
  P^{L}_{\epsilon}= \int _{\epsilon}^{L} \mathrm{d} t \, \mathrm{e}^{-t\Box} = \int _{\epsilon}^{L} \frac{\mathrm{d} t}{(2 \pi t)^{2}} \, \mathrm{e}^{- \left| x-y \right|^{2} / 4t}.
\end{equation*}
Thus, using the heat kernel cutting off, we can rewrite the Feynman diagram with propagator as
\begin{equation*}
  \begin{aligned}
    & \hbar \lambda^{2} \int _{(\mathbb{R}^{4})^{2}} \mathrm{d} ^{4} x_1 \mathrm{d} ^{4} x_2 \, \phi^{2}(x_1) \phi^{2}(x_2) \int _{[\epsilon, L]^{2}} \frac{\mathrm{d} t_1}{(2\pi t_1)^{2}} \frac{\mathrm{d} t_2}{(2\pi t_2)^{2}} \, \mathrm{e}^{- \left| x_1 - x_2 \right|^{2} / 4t_1} \mathrm{e}^{- \left| x_1 - x_2 \right|^{2} / 4t_2} \\
    & = - \hbar \lambda^{2} \frac{\ln \epsilon}{\pi^{2}} \int _{\mathbb{R}^{4}} \mathrm{d} ^{4} x \, \phi^{4}(x) + \text{Smooth Terms}.
  \end{aligned}
\end{equation*}
To cancel the divergence, we need to introduce a counter term into action $ S[\phi]$, which is given by
\begin{equation*}
  S \rightarrow S + I_{1}^{\mathrm{ct}}, \quad
  I_{1}^{\mathrm{ct}} = \frac{\hbar \lambda^{2}}{4! \pi^{2}} \ln \epsilon \int _{\mathbb{R}^{4}} \mathrm{d} ^{4} x \, \phi^{4}(x),
\end{equation*}
% ltex: enabled=false
which could be interpreted as the renormalization of the coupling constant $ \lambda$:
% ltex: enabled=true
\begin{equation*}
  \lambda \rightarrow \lambda + \frac{\hbar \lambda^{2}}{\pi^{2}} \ln \epsilon, \quad \frac{\mathrm{d} \lambda}{\mathrm{d} \ln \epsilon} = \frac{\hbar \lambda^{2}}{\pi^{2}}.
\end{equation*}

\section{Day VII: BV-BRST Formalism}

\subsection{A Toy Model}

First, we consider a finite dimensional gauge theory equipped with a Lagrangian
\begin{equation*}
  f: V \rightarrow \mathbb{C},
\end{equation*}
where $ V $ is a finite dimensional vector space. The gauge redundancy is given by the action of the Lie group $ G$ on $ V$ such that $ f$ is $ G$ invariant.

\subsubsection{Field Space and BRST}

We consider the 'path integral' of this theory, which is given by
\begin{equation*}
  \int _{V/G} \mathrm{d} \mu \, \mathrm{e}^{\mathrm{i}f / \hbar}.
\end{equation*}
There are two approaches to compute this integral:
\begin{enumerate}[(1)]
  \item \textbf{Faddeev-Popov Method}: Find a subspace $ W \subset V$ such that points of $ W$ represents the gauge orbits of $ V$, i.e. $ V/G \cong W$, and construct a form $ \xi_{W}$ associate to $ W$ such that
    \begin{equation*}
      \int _{V/G} \mathrm{d} \mu \, \mathrm{e}^{\mathrm{i}f / \hbar} = \int _{V} \mathrm{d}\mu \, \delta(W) \xi_{W} \mathrm{e}^{\mathrm{i}f / \hbar},
    \end{equation*}
    which is associated with Faddeev-Popov method, where $ W$ is the gauge fixing condition and $ \xi_{W}$ is the action of Faddeev-Popov ghost field.
    This method is very useful in the case of usual gauge theory e.g. Maxwell theory.

    However, some issues would occur while one consider the gauge theory with some singular gauge orbits.
    So we need to introduce the second method, to overcome the issue of singular gauge orbits.

    % borzul thm could be written as a derived form
    In mathematics, we would consider the quotient space by replacing it with the 'derived quotient', which would lead to BV-BRST formalism.

  \item \textbf{BV-BRST Formalism}: Consider the function ring over $ V/G$
    \begin{equation*}
      \mathcal{O}(V/G) = \mathcal{O}(V)^{G}.
    \end{equation*}
    Assume $ G$ is a compact Lie group with Lie algebra $ \mathfrak{g}$, which has a fundamental vector field over $ V$ given by $ X_{u} \in V$, where $ u \in \mathfrak{g}$.
    Thus, the $ G$ invariant function ring could be written as
    \begin{equation*}
      \mathcal{O}(V)^{G} = \left\{ f \in \mathcal{O}(V) \, | \, X_{u} f = 0, \forall u \in \mathfrak{g} \right\}.
    \end{equation*}
    Let $\left\{ e_i \right\}$ be a basis of $ V$, and $\left\{ c^{i} \right\}$ be the dual basis of $ V^{*}$, then we can define the differential operator
    \begin{equation*}
      \mathrm{d} _{\mathrm{CE}} : \phi \mapsto c^{i} \otimes X_{e^{i}} \phi \equiv c^{i} X_{i} \phi,
    \end{equation*}
    where $ X_{i} = X_{e_i}$.
    Thus, the resolution could be written as a complex
    \begin{equation*}
      0 \rightarrow \mathcal{O}(V)^{G} \hookrightarrow \mathcal{O}(V) \xrightarrow{\mathrm{d} _{\mathrm{CE}}} \mathcal{O}(V) \otimes g^{*} \xrightarrow{\mathrm{d} _{\mathrm{CE}}} \mathcal{O}(V) \otimes \bigwedge\nolimits^{2} g^{*} \xrightarrow{\mathrm{d} _{\mathrm{CE}}} \cdots,
    \end{equation*}
    thus, the derived object could be
    \begin{equation*}
      C^{\bullet}(\mathfrak{g}, \mathcal{O}(V)) = \left( \bigwedge\nolimits^{2} \mathfrak{g} \otimes \mathcal{O}(V), \mathrm{d} _{\mathrm{CE}}\right) = \sym\left( (\mathfrak{g}[1])^{*} \right) \oplus \sym(V^{*}),
    \end{equation*}
    which implied that the derived quotient space $ V/G$ could be understood as the function ring over $ \mathfrak{g}^{*} \oplus V$.
    Thus, the path integral over $ V/G$ could be rewritten as
    \begin{equation*}
      \int _{V/G} \mathrm{d} \mu \, \mathrm{e}^{\mathrm{i} f / \hbar} \leadsto \int _{\mathfrak{g}[1] \oplus V} \mathrm{d} \mu \, \mathrm{e}^{\mathrm{i} f / \hbar},
    \end{equation*}
    where $ H^{0}(\mathfrak{g}, \mathcal{O}(V)) = \mathcal{O}(V / G)$. Moreover, if $ G$ acts on $ V$ freely, $ \mathcal{O}(V / G) \cong^{q} C^{\bullet}(\mathfrak{g}, \mathcal{O}(V))$ is a quasi-isomorphism.
    In physics, this is the field space of BV-BRST formalism, where the derived quotient space $ V/G$ is replaced by the derived field space $ \mathfrak{g}[1] \oplus V$, and $ \mathfrak{g}[1]$ is called BRST ghost.
\end{enumerate}

\subsubsection{Integration and BV}

After identify the 'derived field space' $ \mathfrak{g}[1] \oplus V$, we want to consider the integration over it.
However, the highest volume form is not an ideal object to study infinite dimensional integration. The dual of it would be better, which is called the polyvector field
\begin{equation*}
  \mathrm{PV}^{\bullet}(V) = \bigoplus_{p=0}^{\infty}\Gamma\left( V, \wedge^{p} TV \right),
\end{equation*}
equipped with Wedge Product: $\wedge: \mathrm{PV}^{p} \times \mathrm{PV}^{q} \rightarrow \mathrm{PV}^{p+q}$, and the isomorphism between the polyvector field and the de-Rham complex could be written as.
% TODO: add diagram
which leads to the BV operator, which is the generated divergence operator $\Delta_{\omega}$, where $ \omega$ is a volume form on $ V$.
\begin{example}[Path Integral over $ V$ ]
  Consider $ \omega = \mathrm{e}^{\rho(x)} \mathrm{d} x_1 \wedge \cdots \mathrm{d} x_n$, the dual of polyvector field is given by
  \begin{equation*}
    \mathrm{PV}^{\bullet}(V) = C^{\infty }(V)[\partial_{1}, \cdots ,\partial_{n}],
  \end{equation*}
  equipped with the wedge product $ \partial_{i} \wedge \partial_{j} = - \partial_{j} \wedge \partial_{i}$. For connivance, we denote $\partial_{i}$ with wedge product as Grassmann variable $ \theta_{i}$, thus the polyvector field could be written as
  \begin{equation*}
    \mathrm{PV}^{\bullet}(V) \ni \mu = \sum_{I} \mu_{I_1, \cdots ,I_k}(x) \theta_{I_1} \cdots \theta_{I_k}.
  \end{equation*}
  Under the notation above, $ \Delta$ could be written explicitly as
  \begin{equation*}
    \Delta_{\rho} = \frac{\partial }{\partial x^{i}} \frac{\partial }{\partial \theta_{i}} + \partial_{i} \rho(x) \frac{\partial }{\partial \theta_{i}},
  \end{equation*}
  satisfying $ \Delta^{2} = 0$.
\end{example}

The integration over $ V$ could be identified with the linear homomorphism
\begin{equation*}
  H^{d}_{\mathrm{dR}}(V) \leadsto \int : \Omega^{d}(V) \rightarrow \mathbb{C},
\end{equation*}
at the dual side, we can use the language of polyvector field to define the integration over $ V$ as
\begin{equation*}
  H^{0}_{\Delta}(V) \leadsto \int: \mathrm{PV}^{0}(V) \rightarrow \mathbb{C},
\end{equation*}
which is much better in the context of quantum field theory, since it is isolated from the (mostly infinite) dimension of the field space.

\subsection{BV Algebra}

\begin{definition}[BV Algebra]
  A BV algebra is a graded commutative algebra $ (A, \cdot, \Delta, \left\{ ~.~ \right\})$, where
  \begin{enumerate}[(1)]
    \item $ A$ is a graded commutative algebra with $ a \cdot b = (-1)^{\left| a \right|\left| b \right|} b \cdot a$;
    \item (BV Operator) $ \Delta : A \rightarrow A$ is a degree $ 1 $ operator satisfied $ \Delta^{2} = 0$;
    \item (BV Bracket) $\left\{ ~,~ \right\} : A \otimes A \rightarrow A$ is a degree $ 1$ bilinear map
      \begin{equation*}
        \left\{ a, b \right\} = \Delta(a \cdot b) - \Delta(a) \cdot b - (-1)^{\left| a \right|} a \cdot \Delta(b),
      \end{equation*}
  \end{enumerate}
\end{definition}
\begin{proposition}
  BV bracket satisfies the following properties:
  \begin{enumerate}[(1)]
    \item (Graded Symmetry) $\left\{ a,b \right\} = (-1)^{\left| a \right|\left| b \right|} \left\{ b, a \right\}$,
    \item (Graded Leibniz Rule) $\left\{ a, bc \right\} = \left\{ a, b \right\} c + (-1)^{(\left| a \right|+1)\left| b \right|} a \left\{ b,c \right\}$,
    \item (Compatible with BV Operator) $ \Delta\left\{ a,b \right\} = - \left\{ \Delta a, b \right\} - (-1)^{\left| a \right|} \left\{ a, \Delta b \right\}$.
  \end{enumerate}
\end{proposition}
\begin{example}[Polyvector Field as BV Algebra]
  $ A = \left( \mathrm{PV}^{-\bullet}(V), \Delta \right)$, where $ A^{-k} = \mathrm{PV}^{k}(V)$, and BV bracket is Schouten-Nijenhuis bracket (up to sign)
  \begin{equation*}
    \left\{ \alpha, \beta \right\} = (-1)^{\left| \alpha \right|+1} [\alpha, \beta]_{\mathrm{SN}}.
  \end{equation*}
  As we all know, the Schouten-Nijenhuis bracket is associated with the Poisson structure of function ring, thus the process to find the BV operator is the process of quantization of the Poisson algebra, which is associated with deformation quantization.
\end{example}

If we add a derivation $Q: A\rightarrow A$ satisfies $ Q^{2} = 0$, $\deg Q= 1$ and $ Q \Delta + \Delta Q = 0$, then we can define the BV algebra with BRST operator, which is called differential graded BV (DGBV) algebra.

\begin{example}[Calabi-Yau Manifold]
  Let $(X, \Omega)$ is a Calabi-Yau manifold, where $ \Omega$ is the holomorphic volume form. The algebra of polyvector field
  \begin{equation*}
    \mathrm{PV}^{k,l} (X) = \Omega^{0,l}_{\bar{\partial}}(X, \wedge^{k} T^{1,0} X),
  \end{equation*}
  where $ \Omega$ and $\partial$ leads to the BV operator $ \Delta$, and the Dolbeault operator $ \bar{\partial}$ leads to the BRST operator $ Q$, then $\left( \mathrm{PV}^{\bullet, - \bullet}, \bar{\partial}, \Delta \right)$ is a DGBV algebra.
\end{example}

\begin{example}[BRST Ghost and Anti-Field Formalism]
  Consider the gauge theory with gauge group $ G$ and field space $ V$, the algebra of polyvector field is given by
  \begin{equation*}
    A = \sym^{\bullet}(\mathfrak{g}^{*}[-1]) \otimes \sym^{\bullet}(\mathfrak{g}[2]) = \mathbb{R}[c^{\alpha}, e_{\alpha}],
  \end{equation*}
  where $ c^{\alpha}$ is degree $ 1$  basis of $ \mathfrak{g}^{*}$ and $ e_{\alpha}$ is degree $-2$ basis of $ \mathfrak{g}$. The BV operator is given by
  \begin{equation*}
    \Delta = \frac{\partial }{\partial c^{\alpha}} \frac{\partial }{\partial e_{\alpha}}, \quad \Delta^{2} = 0, \quad \deg \Delta = 1,
  \end{equation*}
  thus we can define the BV bracket and form a BV algebra $ A = PV^{\bullet}(\mathfrak{g}[1])$.
  This BV algebra over $(c^{\alpha}, e_{\alpha})$ is an odd version of the BV algebra over $(x^{i}, \theta_{i})$. Which is associated with the anti-field formalism.
\end{example}

Using the BV algebra, we could refactor the field space $ \mathfrak{g}[1] \otimes V$ into a BV algebra $ \mathrm{PV}(\mathfrak{g}[1]) \otimes \mathrm{PV}(V)$, which ensure us to define the integration over the field space i.e. the path integral.

\label{LastPage}
\end{document}
