% !TeX program = pdflatex
\documentclass[10pt]{article}
\input{../preamble.tex}
\fancyhf{}
\fancypagestyle{plain}{
  \lhead{2025}
  \chead{\centering{QFT}}
  \rhead{\thepage\ of \pageref{LastPage}}
  \lfoot{}
  \cfoot{}
\rfoot{}}
\pagestyle{plain}

%-------------------basic info-------------------

\title{\textbf{QFT by Si Li}}
\author{Xinyu Xiang}
\date{Jun. 2025}

%-------------------document---------------------

\begin{document}
\maketitle

\textbf{Warning}: Lots of possible types!!!!!!!!!!!!

\section{Day I: Overall Discussion and Mathematical Preliminaries}

\subsection{Actions and Path Integrals}

Action $ S: \mathcal{E} \rightarrow \mathbb{k}$ where $ \mathcal{E}$ always has infinite dimension, and $ \mathbb{k}$ is a field (usually $ \mathbb{R}$ or $ \mathbb{C}$).

\begin{equation*}
  \text{QM in Imaginary Time} \xrightarrow{\text{ Brownian Motion}} \text{Wiener Measure on Phase Space}
\end{equation*}

\begin{equation*}
  \text{Asymptotic Analysis} \xrightarrow{\quad} \text{Perturbative Renormalisation Theory}
\end{equation*}

\begin{example} Some Examples of Classical Field Theories
  \begin{enumerate}[(a)]
    \item Scalar Field Theory $ \mathcal{E} = C^{\infty }(X)$
    \item Gauge Theory $ \mathcal{E} = \mathrm{Conn}(P,X)$
    \item $ \sigma$ Model $ \mathcal{E} = \text{Maps}(\Sigma, X)$
    \item Gravity $ \mathcal{E} = \text{Metrics}(X)$ (More better descriptions does not depends on the background)
  \end{enumerate}
\end{example}

\subsection{Observables}

Observables are functions on the space of fields, i.e. $ \mathcal{O} \in C^{\infty }(\mathcal{E})$.

\begin{example}[field theory]
  \begin{enumerate}[(a)]
    \item   Consider $ X = pt$, thus $ \mathcal{E} = \mathbb{R}^{n}$ for example.
    \item $ \dim X > 0$, the new algebraic structure arise form topological structures of $ X$.
  \end{enumerate}
\end{example}

The Key Point is: Capture the data of open sets of $ X$ $\longrightarrow$ Consider the observables supported on open set $ U$  of $ X$ denoted by $ \mathrm{Obs}(U)$ where $ U$ is an open set of $ X$.

Local data captures the open sets of $ X$. The relations between open sets captures the global data of $ X$ $\longrightarrow$ The algebraic structure of the observables is a sheaf of $ X$.
\begin{equation*}
  \bigsqcup_{i} U_{i} \longrightarrow \bigotimes_{i} \mathrm{Obs}(U_i)
\end{equation*}
Which implies OPE in physics and factorization algebra in mathematics.

Higher product in QFT: The generalization of products of algebra ('products in any direction instead of left and right') e.g. QM gives only left and right module of an algebra; OPE has products in various directions.

Consider the $\dim X = 2$ case in detailed
\begin{example}[Holomorphic/Chiral Field Theory]
  Various angle of product $ A(w) B(z)$  could be denoted by the time of $ A(w)$ rotations around $ B(z)$, which could be captured by the Fourier mode of $ A(w)$, thus one can have
  \begin{equation*}
    A(w) B(z) = \sum_{m \in \mathbb{Z}} \frac{( A_{(m) B(z)})}{(z - w)^{m+1}}
  \end{equation*}
  which is the Chiral algebra due to Beilinson and Drinfeld and generalized by Gui
\end{example}

\subsection{de Rham Cohomology}

\label{LastPage}
\end{document}
