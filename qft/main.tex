% !TeX program = pdflatex
\documentclass[10pt]{article}
%% ======================================================================
%% LaTeX Preamble for Physics Note, Experiment Report, and Research in
%% CJK and English
%% ======================================================================

%% ----------------------------------------------------------------------
%% 1. 文档基础设置与语言支持
%% ----------------------------------------------------------------------
\usepackage{geometry}
\geometry{a4paper,centering,scale=0.75} % 页面设置

% 中文支持 - 适配 macOS
% \usepackage[UTF8]{ctex}
% \AtBeginDocument{\small} % 全局缩小字体
% \setCJKmainfont[
%   Script=CJK,
%   BoldFont={Heiti SC},
%   ItalicFont={Kaiti SC}
% ]{Songti SC}

%% ----------------------------------------------------------------------
%% 2. 数学支持包
%% ----------------------------------------------------------------------
\usepackage{amsmath,amssymb,amsthm} % 数学基础支持
\usepackage{mathtools} % amsmath 扩展
\usepackage{physics} % 物理公式简化
\usepackage{mathrsfs} % 数学花体字
\usepackage{bbm} % 黑板粗体
\usepackage{cancel} % 公式划除线
\numberwithin{equation}{section} % 按节编号公式
\usepackage[sc]{mathpazo}
\usepackage{courier}
\usepackage{inputenc}
\usepackage[T1]{fontenc}
\usepackage{microtype}
\RequirePackage[font=small,format=plain,labelfont=bf,textfont=it]{caption}

%% ----------------------------------------------------------------------
%% 3. 图形与绘图支持
%% ----------------------------------------------------------------------
\usepackage{graphicx} % 图片支持
\usepackage[export]{adjustbox} % 图片调整
\usepackage{float} % 控制浮动体
\usepackage{pgfplots} % 绘图支持
\pgfplotsset{compat=newest} % 使用最新版本特性
% \usepackage[compat=1.1.0,warn luatex=false]{tikz-feynman} % Feynman 图
\usepackage{quiver} % 交换图

%% ----------------------------------------------------------------------
%% 4. 表格支持
%% ----------------------------------------------------------------------
\usepackage{tabularx} % 增强表格
\usepackage{xltabular} % 长表格支持

%% ----------------------------------------------------------------------
%% 5. 交叉引用与超链接
%% ----------------------------------------------------------------------
% hyperref 通常放在最后加载以避免冲突
\usepackage[bookmarks=true, colorlinks=true, linkcolor=teal,
citecolor=blue, urlcolor=magenta, hidelinks]{hyperref}

%% ----------------------------------------------------------------------
%% 6. 外观与装饰
%% ----------------------------------------------------------------------
\usepackage{fancyhdr} % 页眉页脚
\usepackage[dvipsnames,svgnames]{xcolor} % 扩展颜色支持
\usepackage{framed} % 框架效果
\usepackage{tcolorbox} % 彩色文本框
\tcbuselibrary{most} % tcolorbox 扩展库
\usepackage[strict]{changepage} % 提供 adjustwidth 环境
\usepackage{scalerel} % 缩放支持

%% ----------------------------------------------------------------------
%% 7. 引用与杂项支持
%% ----------------------------------------------------------------------
\usepackage{enumerate} % 列表环境
\usepackage{stackrel} % 符号堆叠
\usepackage{import,xifthen,pdfpages} % 文档导入与条件判断

% 条件加载透明效果包 (仅在 PDF 模式下)
\usepackage{ifpdf}
\ifpdf
\usepackage{transparent}
\else
% 非 PDF 模式下使用替代方案或提供警告
\newcommand{\transparent}[1]{}
\typeout{警告:transparent 包功能在非 PDF 模式下不可用}
\fi

%% ----------------------------------------------------------------------
%% 8. 定理环境设置
%% ----------------------------------------------------------------------
% English theorem environments
\newtheorem{theorem}{Theorem}[section]
\newtheorem{lemma}[theorem]{Lemma}
\newtheorem{proposition}[theorem]{Proposition}
\newtheorem{corollary}[theorem]{Corollary}
\newtheorem{definition}{Definition}[section]
\newtheorem{remark}{Remark}[section]
\newtheorem{example}{Example}[section]
\newtheorem{construction}{Construction}[section]
\newenvironment{observation}{
\begin{proof}[Observation]}{
\end{proof}}
\newenvironment{solution}{
\begin{proof}[Solution]}{
\end{proof}}

%% ----------------------------------------------------------------------
%% 9. 自定义框架和环境
%% ----------------------------------------------------------------------
% 引用块样式
\definecolor{formalshade}{rgb}{0.95,0.95,1}
\newenvironment{quoteblock}{%
  \def\FrameCommand{%
    \hspace{1pt}%
    {\color{DarkBlue}\vrule width 2pt}%
    {\color{formalshade}\vrule width 4pt}%
    \colorbox{formalshade}%
  }%
  \MakeFramed{\advance\hsize-\width\FrameRestore}%
  \noindent\hspace{-4.55pt}%
  \begin{adjustwidth}{}{7pt}%
    \vspace{2pt}\vspace{2pt}%
  }
  {%
    \vspace{2pt}
  \end{adjustwidth}\endMakeFramed%
}

% 解答块样式
\definecolor{brownshade}{rgb}{0.99,0.97,0.93}
\newenvironment{solblock}{%
  \def\FrameCommand{%
    \hspace{1pt}%
    {\color{BurlyWood}\vrule width 2pt}%
    {\color{brownshade}\vrule width 4pt}%
    \colorbox{brownshade}%
  }%
  \MakeFramed{\advance\hsize-\width\FrameRestore}%
  \noindent\hspace{-4.55pt}%
  \begin{adjustwidth}{}{7pt}%
    \vspace{2pt}\vspace{2pt}%
  }
  {%
    \vspace{2pt}
  \end{adjustwidth}\endMakeFramed%
}

% 问题块样式
\definecolor{greenshade}{rgb}{0.90,0.99,0.91}
\newenvironment{quesblock}{%
  \def\FrameCommand{%
    \hspace{0.01pt}%
    {\color{Green}\vrule width 2pt}%
    {\color{greenshade}\vrule width 00.1pt}%
    \colorbox{greenshade}%
  }%
  \MakeFramed{\advance\hsize-\width\FrameRestore}%
  \noindent\hspace{-4.55pt}%
  \begin{adjustwidth}{}{7pt}%
    \vspace{2pt}\vspace{2pt}%
  }
  {%
    \vspace{2pt}
  \end{adjustwidth}\endMakeFramed%
}

% 标记块
\newtcolorbox{markerblock}[1][]{enhanced,
  before skip=2mm,after skip=3mm,
  boxrule=0.4pt,left=5mm,right=2mm,top=1mm,bottom=1mm,
  colback=yellow!20,
  colframe=yellow!40!black,
  sharp corners,rounded corners=southeast,arc is angular,arc=3mm,
  underlay={%
    \path[fill=tcbcolback!80!black] ([yshift=3mm]interior.south east)--++(-0.4,-0.1)--++(0.1,-0.2);
    \path[draw=tcbcolframe,shorten <=-0.05mm,shorten >=-0.05mm] ([yshift=3mm]interior.south east)--++(-0.4,-0.1)--++(0.1,-0.2);
    \path[fill=yellow!80!black,draw=none] (interior.south west) rectangle node[white]{\Huge\bfseries !} ([xshift=4mm]interior.north west);
  },
drop fuzzy shadow,#1}

% 提示块
\definecolor{tipscolor}{rgb}{0.77,0.72,0.65}
\newtcolorbox{tipsblock}[2][]
{enhanced,breakable,
  left=12pt,right=12pt,
  coltitle=white,
  colbacktitle=tipscolor,
  attach boxed title to top left={yshifttext=-1mm},
  boxed title style={skin=enhancedfirst jigsaw,arc=1mm,bottom=0mm,boxrule=0mm},
  boxrule=1pt,
  colback=OldLace,
  colframe=tipscolor,
  sharp corners=northwest,
  title=\vspace{3mm}\textbf{#2},
  arc=1mm,
#1}

% 定理类彩色块
\newenvironment{thmblock}[1][\textbf{Theorem}]{
\begin{tcolorbox}[title=\textbf{#1}, colback=red!5,colframe=red!75!black]}{
\end{tcolorbox}}

\newenvironment{defblock}[1][\textbf{Definition}]{
\begin{tcolorbox}[colback = Emerald!10, colframe = cyan!40!black, title = \textbf{#1}]}{
\end{tcolorbox}}

\newenvironment{lemmablock}[1][\textbf{Lemma}]{
\begin{tcolorbox}[title=\textbf{#1},colback=SeaGreen!10!CornflowerBlue!10,colframe=RoyalPurple!55!Aquamarine!100!]}{
\end{tcolorbox}}

\newenvironment{propblock}[1][\textbf{Proposition}]{
  \begin{tcolorbox}
  [title = \textbf{#1}, colback=Salmon!20, colframe=Salmon!90!Black]}{
\end{tcolorbox}}

\newenvironment{colblock}[1][\textbf{Collary}]{
\begin{tcolorbox}[colback=JungleGreen!10!Cerulean!15,colframe=CornflowerBlue!60!Black,title = \textbf{#1}]}{
\end{tcolorbox}}

% Mathematical operator definitions
\DeclareMathOperator{\im}{im}
\DeclareMathOperator{\coker}{coker}
\DeclareMathOperator{\ind}{ind}

\fancyhf{}
\fancypagestyle{plain}{
  \lhead{2025}
  \chead{\centering{Quantum Field Theory}}
  \rhead{\thepage\ of \pageref{LastPage}}
  \lfoot{}
  \cfoot{}
\rfoot{}}
\pagestyle{plain}

%-------------------basic info-------------------

\title{\textbf{Quantum Field Theory}}
\author{Xinyu Xiang}
\date{Jun. 2025}

%-------------------document---------------------

\begin{document}
\maketitle

\textbf{Warning}: Lots of possible typos!!!!!!!!!!!!
\textbf{Notations}:
\begin{itemize}
  \item $ X$: a smooth manifold, usually a compact manifold.
  \item $ \mathcal{E}$: the space of fields, usually infinite dimensional.
  \item $ \mathrm{Conn}(P,X)$: the space of connections on a principal bundle $ P$ over $ X$.
  \item $ \text{Maps}(\Sigma, X)$: the space of maps from a surface $\Sigma$ to $ X$.
  \item $ \Omega^{\bullet}(X)$: the space of differential forms on $ X$.
  \item $ \Omega^{\bullet}_{c}(X)$: the space of differential forms with compact support on $ X$.
  \item $ \mathrm{Vect}(M)$: the space of smooth vector fields on a manifold $ M$, which is Lie algebra of $ \mathrm{Diff}(M)$.
\end{itemize}

\section{Day I: Overall Discussion and Mathematical Preliminaries}

\subsection{Actions and Path Integrals}

Action $ S: \mathcal{E} \rightarrow \mathbf{k}$ where $ \mathcal{E}$ always has infinite dimension, and $ \mathbb{k}$ is a field (usually $ \mathbb{R}$ or $ \mathbb{C}$).

\begin{equation*}
  \text{QM in Imaginary Time} \xrightarrow{\text{ Brownian Motion}} \text{Wiener Measure on Phase Space}
\end{equation*}

\begin{equation*}
  \text{Asymptotic Analysis} \xrightarrow{\quad} \text{Perturbative Renormalisation Theory}
\end{equation*}

\begin{example} Some Examples of Classical Field Theories
  \begin{enumerate}[(a)]
    \item Scalar Field Theory $ \mathcal{E} = C^{\infty }(X)$
    \item Gauge Theory $ \mathcal{E} = \mathrm{Conn}(P,X)$
    \item $ \sigma$ Model $ \mathcal{E} = \text{Maps}(\Sigma, X)$
    \item Gravity $ \mathcal{E} = \text{Metrics}(X)$ (More better descriptions does not depends on the background)
  \end{enumerate}
\end{example}

\subsection{Observables}

Observables are functions on the space of fields, i.e. $ \mathcal{O} \in C^{\infty }(\mathcal{E})$.

\begin{example}[field theory]
  \begin{enumerate}[(a)]
    \item   Consider $ X = pt$, thus $ \mathcal{E} = \mathbb{R}^{n}$ for example.
    \item $ \dim X > 0$, the new algebraic structure arise form topological structures of $ X$.
  \end{enumerate}
\end{example}

The Key Point is: Capture the data of open sets of $ X$ $\longrightarrow$ Consider the observables supported on open set $ U$  of $ X$ denoted by $ \mathrm{Obs}(U)$ where $ U$ is an open set of $ X$.

Local data captures the open sets of $ X$. The relations between open sets captures the global data of $ X$ $\longrightarrow$ The algebraic structure of the observables is a sheaf of $ X$.
\begin{equation*}
  \bigsqcup_{i} U_{i} \longrightarrow \bigotimes_{i} \mathrm{Obs}(U_i)
\end{equation*}
Which implies OPE in physics and factorization algebra in mathematics.

Higher product in QFT: The generalization of products of algebra ('products in any direction instead of left and right') e.g. QM gives only left and right module of an algebra; OPE has products in various directions.

Consider the $\dim X = 2$ case in detailed
\begin{example}[Holomorphic/Chiral Field Theory]
  Various angle of product $ A(w) B(z)$  could be denoted by the time of $ A(w)$ rotations around $ B(z)$, which could be captured by the Fourier mode of $ A(w)$, thus one can have
  \begin{equation*}
    A(w) B(z) = \sum_{m \in \mathbb{Z}} \frac{( A_{(m) B(z)})}{(z - w)^{m+1}}
  \end{equation*}
  which is the Chiral algebra due to Beilinson and Drinfeld.
\end{example}

\subsection{de Rham Cohomology}

Chain of differential forms $ \Omega^{\bullet}(X)$
\begin{equation}\label{eq:deRhamChain}
  \Omega^{\bullet}(X) = \left( \cdots \xrightarrow{\mathrm{d}} \Omega^{n-1}(X) \xrightarrow{\mathrm{d}} \Omega^{n}(X) \xrightarrow{\mathrm{d}} \Omega^{n+1}(X) \xrightarrow{\mathrm{d}} \cdots \right)
\end{equation}
where $ \mathrm{d} $ is the exterior derivative, and $ \Omega^{n}(X)$ is the space of $ n$-forms on $ X$. The general construction of differential forms could be constructed over open set $ U$ by
\begin{equation*}
  \Omega^{n}(U) = \bigoplus_{1 \le i_1 \le \cdots \le i_n \le n} C^{\infty }(U) \mathrm{d} x^{i_1} \wedge \cdots \wedge \mathrm{d} x^{i_n}
\end{equation*}
where one can prove that $ \mathrm{d} ^{2} = 0$ and thus $\left( \Omega^{\bullet}(U), \mathrm{d} \right)$  is a cochain complex. The cohomology of it is called the de Rham cohomology $ H^{\bullet}(X)$.
\begin{proof}
  TBD.
\end{proof}
\begin{proposition}
  The definition of de Rham cohomology does not depend on the choice of the open set $ U$ and the choice of the coordinate system i.e. it is intrinsic $\longrightarrow$ we can define the de Rham cochain complex on smooth manifold $ X$.
\end{proposition}
\begin{proof}
  TBD.
\end{proof}

\begin{definition}[de Rham Cohomology on Compact Support]
  Let $ X$ be a smooth manifold, then the de Rham cohomology on compact support is defined as
  \begin{equation}
    H^{\bullet}_{c}(X) = H^{\bullet}(\Omega^{\bullet}_{c}(X),\mathrm{d})
  \end{equation}
  where $ \Omega^{\bullet}_{c}(X)$ is the space of differential forms with compact support.
\end{definition}

\begin{theorem}[Stokes' Theorem]
  Let $ X$ be a smooth manifold with boundary, then for any $ \omega \in \Omega^{n}(X)$, we have
  \begin{equation*}
    \int_{X} \mathrm{d} \omega = \int _{\partial X} \omega
  \end{equation*}
  which connects the local data $ \mathrm{d} \Omega^{\bullet}(X)$ and the global data $\partial X$.
\end{theorem}
\begin{theorem}[Poincaré Lemma]
  \begin{equation*}
    H^{p}(\mathbb{R}^{n}) =
    \begin{cases}
      \mathbb{R} & p = 0 \\
      0 & p > 0
    \end{cases}, \quad
    H^{p}_{c}(\mathbb{R}^{n}) =
    \begin{cases}
      0 & p <0 \\
      \mathbb{R} & p = n \\
    \end{cases}
  \end{equation*}
  Generator: $ H^{p}(\mathbb{R}^{n})$ $\rightarrow $ constant function, $ H^{p}_{c}(\mathbb{R}^{n})$ $\rightarrow $ a compact support function $ \alpha = f(x) \mathrm{vol}_{n}$, and $ \int _{\mathbb{R}^{n}} \alpha = 1$.
\end{theorem}
\begin{proof}

\end{proof}
Important: An \emph{Integration} arises from the de Rham cohomology!
\begin{observation}
  \begin{enumerate}[(1)]
    \item if $ \alpha = \mathrm{d} \beta$ where $ \beta \in \Omega_{c}^{n-1}(X)$, then $ \int _{X} \alpha = 0$, thus the generator is $ \alpha$ whose integral is non-zero.
    \item \textbf{Dual Site}: Integration could be captured by the cohomology
      \begin{equation*}
        \int _{\mathbb{R}^{n}} \leftrightarrow H^{n}_{c}(\mathbb{R}^{n}) \cong \mathbb{R}
      \end{equation*}
      Path integral could be interpreted as the integration over $ \mathcal{E}$, which leads to consider the cohomology of it.
  \end{enumerate}
\end{observation}

\subsection{Cartan Formula}

Vector fields could acts on smooth functions via
\begin{equation*}
  V(f) = V^{i} \frac{\partial f}{\partial x^{i}} = \frac{\mathrm{d} }{\mathrm{d} t} f(\varphi_{t}(x)) \bigg|_{t=0} = \frac{\mathrm{d} }{\mathrm{d} t} \varphi_{t}^{*} f(x) \bigg|_{t=0}
\end{equation*}
Such an action could be extended to differential forms by
\begin{equation*}
  \mathrm{Vect}(M) \ni V : \alpha \mapsto \mathcal{L}_{V} \alpha = \frac{\mathrm{d} }{\mathrm{d} t} \varphi_{t}^{*} \alpha \bigg|_{t=0}
\end{equation*}
which has the property $ \mathcal{L}_{V}(\alpha \wedge \beta) = \mathcal{L}_{V} \alpha \wedge \beta + \alpha \wedge \mathcal{L}_{V} \beta$, which implies that the Lie derivative is a derivation on the algebra of differential forms with degree $ 0$. And we have contraction $ \iota_{V}$ which is a derivation of degree $ -1$ on the algebra of differential forms.
\begin{equation*}
  \mathcal{L}_{V}  = \mathrm{d} \iota_{V} + \iota_{V} \mathrm{d}
\end{equation*}
Lie derivative is homotopy trivial i.e. chain homotopic.

Rescaling invariance of $ \mathbb{R}^{n}$ leads to the Euler vector field $ E = x^{i} \frac{\partial }{\partial x^{i}}$.

\label{LastPage}
\end{document}
