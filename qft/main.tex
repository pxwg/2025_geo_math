% !TeX program = pdflatex
\documentclass[10pt]{article}
\input{../preamble.tex}
\fancyhf{}
\fancypagestyle{plain}{
  \lhead{2025}
  \chead{\centering{QFT}}
  \rhead{\thepage\ of \pageref{LastPage}}
  \lfoot{}
  \cfoot{}
\rfoot{}}
\pagestyle{plain}

%-------------------basic info-------------------

\title{\textbf{QFT}}
\author{Xinyu Xiang}
\date{Jun. 2025}

%-------------------document---------------------

\begin{document}
\maketitle

\textbf{Warning}: Lots of possible typos!!!!!!!!!!!!
\textbf{Notations}:
\begin{itemize}
  \item $ X$: a smooth manifold, usually a compact manifold.
  \item $ \mathcal{E}$: the space of fields, usually infinite dimensional.
  \item $ \mathbf{k}$: a field, usually $ \mathbb{R}$ or $ \mathbb{C}$.
  \item $ \mathrm{Conn}(P,X)$: the space of connections on a principal bundle $ P$ over $ X$.
  \item $ \text{Maps}(\Sigma, X)$: the space of maps from a surface $\Sigma$ to $ X$.
  \item $ \Omega^{\bullet}(X)$: the space of differential forms on $ X$.
  \item $ \Omega^{\bullet}_{c}(X)$: the space of differential forms with compact support on $ X$.
  \item $ \mathrm{Vect}(M)$: the space of smooth vector fields on a manifold $ M$, which is Lie algebra of $ \mathrm{Diff}(M)$.
\end{itemize}

\section{Day I: Overall Discussion and Mathematical Preliminaries}

\subsection{Actions and Path Integrals}

Action $ S: \mathcal{E} \rightarrow \mathbf{k}$ where $ \mathcal{E}$ always has infinite dimension, and $ \mathbb{k}$ is a field (usually $ \mathbb{R}$ or $ \mathbb{C}$).

\begin{equation*}
  \text{QM in Imaginary Time} \xrightarrow{\text{ Brownian Motion}} \text{Wiener Measure on Phase Space}
\end{equation*}

\begin{equation*}
  \text{Asymptotic Analysis} \xrightarrow{\quad} \text{Perturbative Renormalisation Theory}
\end{equation*}

\begin{example} Some Examples of Classical Field Theories
  \begin{enumerate}[(a)]
    \item Scalar Field Theory $ \mathcal{E} = C^{\infty }(X)$
    \item Gauge Theory $ \mathcal{E} = \mathrm{Conn}(P,X)$
    \item $ \sigma$ Model $ \mathcal{E} = \text{Maps}(\Sigma, X)$
    \item Gravity $ \mathcal{E} = \text{Metrics}(X)$ (More better descriptions does not depends on the background)
  \end{enumerate}
\end{example}

\subsection{Observables}

Observables are functions on the space of fields, i.e. $ \mathcal{O} \in C^{\infty }(\mathcal{E})$.

\begin{example}[field theory]
  \begin{enumerate}[(a)]
    \item   Consider $ X = pt$, thus $ \mathcal{E} = \mathbb{R}^{n}$ for example.
    \item $ \dim X > 0$, the new algebraic structure arise form topological structures of $ X$.
  \end{enumerate}
\end{example}

The Key Point is: Capture the data of open sets of $ X$ $\longrightarrow$ Consider the observables supported on open set $ U$  of $ X$ denoted by $ \mathrm{Obs}(U)$ where $ U$ is an open set of $ X$.

Local data captures the open sets of $ X$. The relations between open sets captures the global data of $ X$ $\longrightarrow$ The algebraic structure of the observables is a sheaf of $ X$.
\begin{equation*}
  \bigsqcup_{i} U_{i} \longrightarrow \bigotimes_{i} \mathrm{Obs}(U_i)
\end{equation*}
Which implies OPE in physics and factorization algebra in mathematics.

Higher product in QFT: The generalization of products of algebra ('products in any direction instead of left and right') e.g. QM gives only left and right module of an algebra; OPE has products in various directions.

Consider the $\dim X = 2$ case in detailed
\begin{example}[Holomorphic/Chiral Field Theory]
  Various angle of product $ A(w) B(z)$  could be denoted by the time of $ A(w)$ rotations around $ B(z)$, which could be captured by the Fourier mode of $ A(w)$, thus one can have
  \begin{equation*}
    A(w) B(z) = \sum_{m \in \mathbb{Z}} \frac{( A_{(m) B(z)})}{(z - w)^{m+1}}
  \end{equation*}
  which is the Chiral algebra due to Beilinson and Drinfeld.
\end{example}

\subsection{de Rham Cohomology}

Chain of differential forms $ \Omega^{\bullet}(X)$
\begin{equation}\label{eq:deRhamChain}
  \Omega^{\bullet}(X) = \left( \cdots \xrightarrow{\mathrm{d}} \Omega^{n-1}(X) \xrightarrow{\mathrm{d}} \Omega^{n}(X) \xrightarrow{\mathrm{d}} \Omega^{n+1}(X) \xrightarrow{\mathrm{d}} \cdots \right)
\end{equation}
where $ \mathrm{d} $ is the exterior derivative, and $ \Omega^{n}(X)$ is the space of $ n$-forms on $ X$. The general construction of differential forms could be constructed over open set $ U$ by
\begin{equation*}
  \Omega^{n}(U) = \bigoplus_{1 \le i_1 \le \cdots \le i_n \le n} C^{\infty }(U) \mathrm{d} x^{i_1} \wedge \cdots \wedge \mathrm{d} x^{i_n}
\end{equation*}
where one can prove that $ \mathrm{d} ^{2} = 0$ and thus $\left( \Omega^{\bullet}(U), \mathrm{d} \right)$  is a cochain complex. The cohomology of it is called the de Rham cohomology $ H^{\bullet}(X)$.
\begin{proof}
  TBD.
\end{proof}
\begin{proposition}
  The definition of de Rham cohomology does not depend on the choice of the open set $ U$ and the choice of the coordinate system i.e. it is intrinsic $\longrightarrow$ we can define the de Rham cochain complex on smooth manifold $ X$.
\end{proposition}
\begin{proof}
  TBD.
\end{proof}

\begin{definition}[de Rham Cohomology on Compact Support]
  Let $ X$ be a smooth manifold, then the de Rham cohomology on compact support is defined as
  \begin{equation}
    H^{\bullet}_{c}(X) = H^{\bullet}(\Omega^{\bullet}_{c}(X),\mathrm{d})
  \end{equation}
  where $ \Omega^{\bullet}_{c}(X)$ is the space of differential forms with compact support.
\end{definition}

\begin{theorem}[Stokes' Theorem]
  Let $ X$ be a smooth manifold with boundary, then for any $ \omega \in \Omega^{n}(X)$, we have
  \begin{equation*}
    \int_{X} \mathrm{d} \omega = \int _{\partial X} \omega
  \end{equation*}
  which connects the local data $ \mathrm{d} \Omega^{\bullet}(X)$ and the global data $\partial X$.
\end{theorem}
\begin{theorem}[Poincaré Lemma]
  \begin{equation*}
    H^{p}(\mathbb{R}^{n}) =
    \begin{cases}
      \mathbb{R} & p = 0 \\
      0 & p > 0
    \end{cases}, \quad
    H^{p}_{c}(\mathbb{R}^{n}) =
    \begin{cases}
      0 & p <0 \\
      \mathbb{R} & p = n \\
    \end{cases}
  \end{equation*}
  Generator: $ H^{p}(\mathbb{R}^{n})$ $\rightarrow $ constant function, $ H^{p}_{c}(\mathbb{R}^{n})$ $\rightarrow $ a compact support function $ \alpha = f(x) \mathrm{vol}_{n}$, and $ \int _{\mathbb{R}^{n}} \alpha = 1$.
\end{theorem}
\begin{proof}

\end{proof}
Important: An \emph{Integration} arises from the de Rham cohomology!
\begin{observation}
  \begin{enumerate}[(1)]
    \item if $ \alpha = \mathrm{d} \beta$ where $ \beta \in \Omega_{c}^{n-1}(X)$, then $ \int _{X} \alpha = 0$, thus the generator is $ \alpha$ whose integral is non-zero.
    \item \textbf{Dual Site}: Integration could be captured by the cohomology
      \begin{equation*}
        \int _{\mathbb{R}^{n}} \leftrightarrow H^{n}_{c}(\mathbb{R}^{n}) \cong \mathbb{R}
      \end{equation*}
      Path integral could be interpreted as the integration over $ \mathcal{E}$, which leads to consider the cohomology of it.
  \end{enumerate}
\end{observation}

\subsection{Cartan Formula}

Vector fields could acts on smooth functions via
\begin{equation*}
  V(f) = V^{i} \frac{\partial f}{\partial x^{i}} = \frac{\mathrm{d} }{\mathrm{d} t} f(\varphi_{t}(x)) \bigg|_{t=0} = \frac{\mathrm{d} }{\mathrm{d} t} \varphi_{t}^{*} f(x) \bigg|_{t=0}
\end{equation*}
Such an action could be extended to differential forms by
\begin{equation*}
  \mathrm{Vect}(M) \ni V : \alpha \mapsto \mathcal{L}_{V} \alpha = \frac{\mathrm{d} }{\mathrm{d} t} \varphi_{t}^{*} \alpha \bigg|_{t=0}
\end{equation*}
which has the property $ \mathcal{L}_{V}(\alpha \wedge \beta) = \mathcal{L}_{V} \alpha \wedge \beta + \alpha \wedge \mathcal{L}_{V} \beta$, which implies that the Lie derivative is a derivation on the algebra of differential forms with degree $ 0$. And we have contraction $ \iota_{V}$ which is a derivation of degree $ -1$ on the algebra of differential forms.
\begin{equation*}
  \mathcal{L}_{V}  = \mathrm{d} \iota_{V} + \iota_{V} \mathrm{d}
\end{equation*}
Lie derivative is homotopy trivial i.e. chain homotopic.

Rescaling invariance of $ \mathbb{R}^{n}$ leads to the Euler vector field $ E = x^{i} \frac{\partial }{\partial x^{i}}$.

\label{LastPage}
\end{document}
